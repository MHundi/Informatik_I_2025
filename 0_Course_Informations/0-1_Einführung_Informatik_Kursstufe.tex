% Options for packages loaded elsewhere
\PassOptionsToPackage{unicode}{hyperref}
\PassOptionsToPackage{hyphens}{url}
\PassOptionsToPackage{dvipsnames,svgnames,x11names}{xcolor}
%
\documentclass[
  11pt,
  a4paper,
  DIV=11,
  numbers=noendperiod]{scrartcl}

\usepackage{amsmath,amssymb}
\usepackage{iftex}
\ifPDFTeX
  \usepackage[T1]{fontenc}
  \usepackage[utf8]{inputenc}
  \usepackage{textcomp} % provide euro and other symbols
\else % if luatex or xetex
  \usepackage{unicode-math}
  \defaultfontfeatures{Scale=MatchLowercase}
  \defaultfontfeatures[\rmfamily]{Ligatures=TeX,Scale=1}
\fi
\usepackage{lmodern}
\ifPDFTeX\else  
    % xetex/luatex font selection
    \setmainfont[]{Avenir Next}
\fi
% Use upquote if available, for straight quotes in verbatim environments
\IfFileExists{upquote.sty}{\usepackage{upquote}}{}
\IfFileExists{microtype.sty}{% use microtype if available
  \usepackage[]{microtype}
  \UseMicrotypeSet[protrusion]{basicmath} % disable protrusion for tt fonts
}{}
\makeatletter
\@ifundefined{KOMAClassName}{% if non-KOMA class
  \IfFileExists{parskip.sty}{%
    \usepackage{parskip}
  }{% else
    \setlength{\parindent}{0pt}
    \setlength{\parskip}{6pt plus 2pt minus 1pt}}
}{% if KOMA class
  \KOMAoptions{parskip=half}}
\makeatother
\usepackage{xcolor}
\usepackage[top=3cm, bottom=2cm, left=4cm, right=2cm]{geometry}
\setlength{\emergencystretch}{3em} % prevent overfull lines
\setcounter{secnumdepth}{-\maxdimen} % remove section numbering
% Make \paragraph and \subparagraph free-standing
\makeatletter
\ifx\paragraph\undefined\else
  \let\oldparagraph\paragraph
  \renewcommand{\paragraph}{
    \@ifstar
      \xxxParagraphStar
      \xxxParagraphNoStar
  }
  \newcommand{\xxxParagraphStar}[1]{\oldparagraph*{#1}\mbox{}}
  \newcommand{\xxxParagraphNoStar}[1]{\oldparagraph{#1}\mbox{}}
\fi
\ifx\subparagraph\undefined\else
  \let\oldsubparagraph\subparagraph
  \renewcommand{\subparagraph}{
    \@ifstar
      \xxxSubParagraphStar
      \xxxSubParagraphNoStar
  }
  \newcommand{\xxxSubParagraphStar}[1]{\oldsubparagraph*{#1}\mbox{}}
  \newcommand{\xxxSubParagraphNoStar}[1]{\oldsubparagraph{#1}\mbox{}}
\fi
\makeatother


\providecommand{\tightlist}{%
  \setlength{\itemsep}{0pt}\setlength{\parskip}{0pt}}\usepackage{longtable,booktabs,array}
\usepackage{calc} % for calculating minipage widths
% Correct order of tables after \paragraph or \subparagraph
\usepackage{etoolbox}
\makeatletter
\patchcmd\longtable{\par}{\if@noskipsec\mbox{}\fi\par}{}{}
\makeatother
% Allow footnotes in longtable head/foot
\IfFileExists{footnotehyper.sty}{\usepackage{footnotehyper}}{\usepackage{footnote}}
\makesavenoteenv{longtable}
\usepackage{graphicx}
\makeatletter
\newsavebox\pandoc@box
\newcommand*\pandocbounded[1]{% scales image to fit in text height/width
  \sbox\pandoc@box{#1}%
  \Gscale@div\@tempa{\textheight}{\dimexpr\ht\pandoc@box+\dp\pandoc@box\relax}%
  \Gscale@div\@tempb{\linewidth}{\wd\pandoc@box}%
  \ifdim\@tempb\p@<\@tempa\p@\let\@tempa\@tempb\fi% select the smaller of both
  \ifdim\@tempa\p@<\p@\scalebox{\@tempa}{\usebox\pandoc@box}%
  \else\usebox{\pandoc@box}%
  \fi%
}
% Set default figure placement to htbp
\def\fps@figure{htbp}
\makeatother

\usepackage[document]{ragged2e}
\KOMAoption{captions}{tableheading}
\makeatletter
\@ifpackageloaded{caption}{}{\usepackage{caption}}
\AtBeginDocument{%
\ifdefined\contentsname
  \renewcommand*\contentsname{Table of contents}
\else
  \newcommand\contentsname{Table of contents}
\fi
\ifdefined\listfigurename
  \renewcommand*\listfigurename{List of Figures}
\else
  \newcommand\listfigurename{List of Figures}
\fi
\ifdefined\listtablename
  \renewcommand*\listtablename{List of Tables}
\else
  \newcommand\listtablename{List of Tables}
\fi
\ifdefined\figurename
  \renewcommand*\figurename{Figure}
\else
  \newcommand\figurename{Figure}
\fi
\ifdefined\tablename
  \renewcommand*\tablename{Table}
\else
  \newcommand\tablename{Table}
\fi
}
\@ifpackageloaded{float}{}{\usepackage{float}}
\floatstyle{ruled}
\@ifundefined{c@chapter}{\newfloat{codelisting}{h}{lop}}{\newfloat{codelisting}{h}{lop}[chapter]}
\floatname{codelisting}{Listing}
\newcommand*\listoflistings{\listof{codelisting}{List of Listings}}
\makeatother
\makeatletter
\makeatother
\makeatletter
\@ifpackageloaded{caption}{}{\usepackage{caption}}
\@ifpackageloaded{subcaption}{}{\usepackage{subcaption}}
\makeatother

\usepackage{bookmark}

\IfFileExists{xurl.sty}{\usepackage{xurl}}{} % add URL line breaks if available
\urlstyle{same} % disable monospaced font for URLs
\hypersetup{
  pdftitle={0.1 Informatik in der Kursstufe},
  colorlinks=true,
  linkcolor={blue},
  filecolor={Maroon},
  citecolor={Blue},
  urlcolor={Blue},
  pdfcreator={LaTeX via pandoc}}


\title{0.1 Informatik in der Kursstufe}
\author{}
\date{}

\begin{document}
\maketitle


\subsection{Bewertungsgrundlagen}\label{bewertungsgrundlagen}

\begin{itemize}
\tightlist
\item
  Eine Klausur pro Semester
\item
  eventuell GFS
\item
  schriftliche Leistungen: 60\%
\item
  unterrichtspraktische Leistungen: 40\% = mündliche Leistungen +
  Programmierleistung + Mitarbeit
\end{itemize}

\subsection{Klausuren}\label{klausuren}

\subsubsection{Korrekturzeichen}\label{korrekturzeichen}

\begin{longtable}[]{@{}cl@{}}
\toprule\noalign{}
Korrekturzeichen & Bedeutung \\
\midrule\noalign{}
\endhead
\bottomrule\noalign{}
\endlastfoot
D & Denkfehler \\
FS & Fachsprache \\
Log & Logikfehler \\
R & Rechenfehler \\
S & Schreibfehler \\
Syn & Syntaxfehler \\
uv & unvollständig \\
Vz & Vorzeichenfehler \\
ug & ungenau \\
r & richtiges Zwischenergebenis \\
(r) & richtiges Zwischenergebnis nach Fehler \\
\end{longtable}

\subsubsection{sprachliche
Korrekturzeichen}\label{sprachliche-korrekturzeichen}

\begin{longtable}[]{@{}cl@{}}
\toprule\noalign{}
Korrekturzeichen & Bedeutung \\
\midrule\noalign{}
\endhead
\bottomrule\noalign{}
\endlastfoot
Gr & Grammatikfehler \\
Rs & Rechtschreibfehler \\
Z & Zeichensetzungsfehler \\
\end{longtable}

\subsubsection{Notentabelle}\label{notentabelle}

\begin{longtable}[]{@{}cccc@{}}
\toprule\noalign{}
Bewertungseinheiten (\%) & Bewertungseinheiten (BE) & Punkte & Note \\
\midrule\noalign{}
\endhead
\bottomrule\noalign{}
\endlastfoot
95-100 & 120-114 & 15 & 1+ \\
90-94 & 113-108 & 14 & 1 \\
85-89 & 107-102 & 13 & 1- \\
80-84 & 101-96 & 12 & 2+ \\
75-79 & 95-90 & 11 & 2 \\
70-74 & 89-84 & 10 & 2- \\
65-69 & 83-78 & 9 & 3+ \\
60-64 & 77-72 & 8 & 3 \\
55-59 & 71-66 & 7 & 3- \\
50-54 & 65-60 & 6 & 4+ \\
45-49 & 59-54 & 5 & 4 \\
40-44 & 53-48 & 4 & 4- \\
33-39 & 47-40 & 3 & 5+ \\
27-32 & 39-32 & 2 & 5 \\
20-26 & 31-24 & 1 & 5- \\
0-19 & 23-0 & 0 & 6 \\
\end{longtable}

\subsubsection{Operatoren}\label{operatoren}

\begin{longtable}[]{@{}
  >{\raggedright\arraybackslash}p{(\linewidth - 4\tabcolsep) * \real{0.2273}}
  >{\raggedright\arraybackslash}p{(\linewidth - 4\tabcolsep) * \real{0.3182}}
  >{\centering\arraybackslash}p{(\linewidth - 4\tabcolsep) * \real{0.4545}}@{}}
\toprule\noalign{}
\begin{minipage}[b]{\linewidth}\raggedright
Operator
\end{minipage} & \begin{minipage}[b]{\linewidth}\raggedright
Beschreibung
\end{minipage} & \begin{minipage}[b]{\linewidth}\centering
Anforderungsbereich
\end{minipage} \\
\midrule\noalign{}
\endhead
\bottomrule\noalign{}
\endlastfoot
\textbf{analysieren} & Strukturen und Zusammenhänge herausarbeiten und
darstellen & II \\
\textbf{angeben} & Sachverhalte ohne Erläuterung auflisten & I \\
\textbf{begründen} & Aussagen durch Argumente stützen & II \\
\textbf{beschreiben} & Sachverhalte strukturiert wiedergeben & I \\
\textbf{beurteilen} & Aussagen/Verfahren an Kriterien messen und
Ergebnis formulieren & III \\
\textbf{bewerten} & Eigenes Urteil zu einem Sachverhalt formulieren und
begründen & III \\
\textbf{darstellen} & Sachverhalte strukturiert wiedergeben & I \\
\textbf{dokumentieren} & Alle relevanten Erklärungen, Herleitungen und
Skizzen angeben & II \\
\textbf{entscheiden} & Begründete Auswahl zwischen Alternativen treffen
& III \\
\textbf{entwerfen} & Konzept oder Modell zu vorgegebenen Kriterien
entwickeln & III \\
\textbf{entwickeln} & Schrittweise Herleitung einer Lösung oder eines
Programms & III \\
\textbf{erklären} & Sachverhalte durch Wissen und Einsichten
verständlich machen & II \\
\textbf{erläutern} & Sachverhalte veranschaulichen und durch Beispiele
verdeutlichen & II \\
\textbf{ermitteln} & Sachverhalte durch Berechnung oder Interpretation
bestimmen & II \\
\textbf{erstellen} & Programm, Diagramm oder Tabelle anfertigen & II \\
\textbf{implementieren} & Algorithmus in einer Programmiersprache
umsetzen & II \\
\textbf{modellieren} & Realen Sachverhalt in formaler Darstellung
beschreiben & III \\
\textbf{nennen} & Sachverhalte ohne Erläuterung angeben & I \\
\textbf{prüfen} & Aussagen oder Lösungsansätze auf Korrektheit
untersuchen & II \\
\textbf{vergleichen} & Gemeinsamkeiten und Unterschiede herausarbeiten &
II \\
\textbf{vervollständigen} & Unvollständige Angaben sinnvoll ergänzen &
II \\
\end{longtable}

\subsection{GFS}\label{gfs}

\begin{itemize}
\tightlist
\item
  Anmeldung immer zu Halbjahresbeginn
\item
  Vortrag 10 bis 15 Minuten
\item
  eventuelle 5 Minuten Übungen
\item
  10 Minuten Kolloquium und Übungen
\item
  10 Minuten Kolloquium
\item
  Nur eine GFS pro Person
\end{itemize}

\subsection{Unterrichtspraktische
Leistungen}\label{unterrichtspraktische-leistungen}

\begin{itemize}
\tightlist
\item
  Fragen
\item
  Antworten
\item
  Aufgabenkontrolle
\item
  Mitarbeit
\item
  Programmierleistung
\end{itemize}

\subsubsection{Obligatorische
Materialien}\label{obligatorische-materialien}

\begin{itemize}
\tightlist
\item
  Heft oder Ordner mit karierten Blättern für Notizen
\end{itemize}

\subsubsection{Inhalte}\label{inhalte}

\begin{enumerate}
\def\labelenumi{\arabic{enumi}.}
\tightlist
\item
  Informatik I: Einführung in die Programmierung mit Python
\item
  Informatik II: Algorithmen
\item
  Informatik III: Theoretische Informatik, A.I. und neuronale Netze
\item
  Informatik IV: Datenbanken
\end{enumerate}

\subsubsection{Dokumente auf GitHub}\label{dokumente-auf-github}

Die Skripte sind zu finden unter

https://github.com/MHundi/Informatik\_I\_2025




\end{document}
