% Options for packages loaded elsewhere
\PassOptionsToPackage{unicode}{hyperref}
\PassOptionsToPackage{hyphens}{url}
\PassOptionsToPackage{dvipsnames,svgnames,x11names}{xcolor}
%
\documentclass[
  11pt,
  a4paper,
  DIV=11,
  numbers=noendperiod]{scrartcl}

\usepackage{amsmath,amssymb}
\usepackage{iftex}
\ifPDFTeX
  \usepackage[T1]{fontenc}
  \usepackage[utf8]{inputenc}
  \usepackage{textcomp} % provide euro and other symbols
\else % if luatex or xetex
  \usepackage{unicode-math}
  \defaultfontfeatures{Scale=MatchLowercase}
  \defaultfontfeatures[\rmfamily]{Ligatures=TeX,Scale=1}
\fi
\usepackage{lmodern}
\ifPDFTeX\else  
    % xetex/luatex font selection
    \setmainfont[]{Avenir Next}
\fi
% Use upquote if available, for straight quotes in verbatim environments
\IfFileExists{upquote.sty}{\usepackage{upquote}}{}
\IfFileExists{microtype.sty}{% use microtype if available
  \usepackage[]{microtype}
  \UseMicrotypeSet[protrusion]{basicmath} % disable protrusion for tt fonts
}{}
\makeatletter
\@ifundefined{KOMAClassName}{% if non-KOMA class
  \IfFileExists{parskip.sty}{%
    \usepackage{parskip}
  }{% else
    \setlength{\parindent}{0pt}
    \setlength{\parskip}{6pt plus 2pt minus 1pt}}
}{% if KOMA class
  \KOMAoptions{parskip=half}}
\makeatother
\usepackage{xcolor}
\setlength{\emergencystretch}{3em} % prevent overfull lines
\setcounter{secnumdepth}{-\maxdimen} % remove section numbering
% Make \paragraph and \subparagraph free-standing
\makeatletter
\ifx\paragraph\undefined\else
  \let\oldparagraph\paragraph
  \renewcommand{\paragraph}{
    \@ifstar
      \xxxParagraphStar
      \xxxParagraphNoStar
  }
  \newcommand{\xxxParagraphStar}[1]{\oldparagraph*{#1}\mbox{}}
  \newcommand{\xxxParagraphNoStar}[1]{\oldparagraph{#1}\mbox{}}
\fi
\ifx\subparagraph\undefined\else
  \let\oldsubparagraph\subparagraph
  \renewcommand{\subparagraph}{
    \@ifstar
      \xxxSubParagraphStar
      \xxxSubParagraphNoStar
  }
  \newcommand{\xxxSubParagraphStar}[1]{\oldsubparagraph*{#1}\mbox{}}
  \newcommand{\xxxSubParagraphNoStar}[1]{\oldsubparagraph{#1}\mbox{}}
\fi
\makeatother

\usepackage{color}
\usepackage{fancyvrb}
\newcommand{\VerbBar}{|}
\newcommand{\VERB}{\Verb[commandchars=\\\{\}]}
\DefineVerbatimEnvironment{Highlighting}{Verbatim}{commandchars=\\\{\}}
% Add ',fontsize=\small' for more characters per line
\usepackage{framed}
\definecolor{shadecolor}{RGB}{241,243,245}
\newenvironment{Shaded}{\begin{snugshade}}{\end{snugshade}}
\newcommand{\AlertTok}[1]{\textcolor[rgb]{0.68,0.00,0.00}{#1}}
\newcommand{\AnnotationTok}[1]{\textcolor[rgb]{0.37,0.37,0.37}{#1}}
\newcommand{\AttributeTok}[1]{\textcolor[rgb]{0.40,0.45,0.13}{#1}}
\newcommand{\BaseNTok}[1]{\textcolor[rgb]{0.68,0.00,0.00}{#1}}
\newcommand{\BuiltInTok}[1]{\textcolor[rgb]{0.00,0.23,0.31}{#1}}
\newcommand{\CharTok}[1]{\textcolor[rgb]{0.13,0.47,0.30}{#1}}
\newcommand{\CommentTok}[1]{\textcolor[rgb]{0.37,0.37,0.37}{#1}}
\newcommand{\CommentVarTok}[1]{\textcolor[rgb]{0.37,0.37,0.37}{\textit{#1}}}
\newcommand{\ConstantTok}[1]{\textcolor[rgb]{0.56,0.35,0.01}{#1}}
\newcommand{\ControlFlowTok}[1]{\textcolor[rgb]{0.00,0.23,0.31}{\textbf{#1}}}
\newcommand{\DataTypeTok}[1]{\textcolor[rgb]{0.68,0.00,0.00}{#1}}
\newcommand{\DecValTok}[1]{\textcolor[rgb]{0.68,0.00,0.00}{#1}}
\newcommand{\DocumentationTok}[1]{\textcolor[rgb]{0.37,0.37,0.37}{\textit{#1}}}
\newcommand{\ErrorTok}[1]{\textcolor[rgb]{0.68,0.00,0.00}{#1}}
\newcommand{\ExtensionTok}[1]{\textcolor[rgb]{0.00,0.23,0.31}{#1}}
\newcommand{\FloatTok}[1]{\textcolor[rgb]{0.68,0.00,0.00}{#1}}
\newcommand{\FunctionTok}[1]{\textcolor[rgb]{0.28,0.35,0.67}{#1}}
\newcommand{\ImportTok}[1]{\textcolor[rgb]{0.00,0.46,0.62}{#1}}
\newcommand{\InformationTok}[1]{\textcolor[rgb]{0.37,0.37,0.37}{#1}}
\newcommand{\KeywordTok}[1]{\textcolor[rgb]{0.00,0.23,0.31}{\textbf{#1}}}
\newcommand{\NormalTok}[1]{\textcolor[rgb]{0.00,0.23,0.31}{#1}}
\newcommand{\OperatorTok}[1]{\textcolor[rgb]{0.37,0.37,0.37}{#1}}
\newcommand{\OtherTok}[1]{\textcolor[rgb]{0.00,0.23,0.31}{#1}}
\newcommand{\PreprocessorTok}[1]{\textcolor[rgb]{0.68,0.00,0.00}{#1}}
\newcommand{\RegionMarkerTok}[1]{\textcolor[rgb]{0.00,0.23,0.31}{#1}}
\newcommand{\SpecialCharTok}[1]{\textcolor[rgb]{0.37,0.37,0.37}{#1}}
\newcommand{\SpecialStringTok}[1]{\textcolor[rgb]{0.13,0.47,0.30}{#1}}
\newcommand{\StringTok}[1]{\textcolor[rgb]{0.13,0.47,0.30}{#1}}
\newcommand{\VariableTok}[1]{\textcolor[rgb]{0.07,0.07,0.07}{#1}}
\newcommand{\VerbatimStringTok}[1]{\textcolor[rgb]{0.13,0.47,0.30}{#1}}
\newcommand{\WarningTok}[1]{\textcolor[rgb]{0.37,0.37,0.37}{\textit{#1}}}

\providecommand{\tightlist}{%
  \setlength{\itemsep}{0pt}\setlength{\parskip}{0pt}}\usepackage{longtable,booktabs,array}
\usepackage{calc} % for calculating minipage widths
% Correct order of tables after \paragraph or \subparagraph
\usepackage{etoolbox}
\makeatletter
\patchcmd\longtable{\par}{\if@noskipsec\mbox{}\fi\par}{}{}
\makeatother
% Allow footnotes in longtable head/foot
\IfFileExists{footnotehyper.sty}{\usepackage{footnotehyper}}{\usepackage{footnote}}
\makesavenoteenv{longtable}
\usepackage{graphicx}
\makeatletter
\newsavebox\pandoc@box
\newcommand*\pandocbounded[1]{% scales image to fit in text height/width
  \sbox\pandoc@box{#1}%
  \Gscale@div\@tempa{\textheight}{\dimexpr\ht\pandoc@box+\dp\pandoc@box\relax}%
  \Gscale@div\@tempb{\linewidth}{\wd\pandoc@box}%
  \ifdim\@tempb\p@<\@tempa\p@\let\@tempa\@tempb\fi% select the smaller of both
  \ifdim\@tempa\p@<\p@\scalebox{\@tempa}{\usebox\pandoc@box}%
  \else\usebox{\pandoc@box}%
  \fi%
}
% Set default figure placement to htbp
\def\fps@figure{htbp}
\makeatother

\usepackage[document]{ragged2e}
\KOMAoption{captions}{tableheading}
\makeatletter
\@ifpackageloaded{caption}{}{\usepackage{caption}}
\AtBeginDocument{%
\ifdefined\contentsname
  \renewcommand*\contentsname{Table of contents}
\else
  \newcommand\contentsname{Table of contents}
\fi
\ifdefined\listfigurename
  \renewcommand*\listfigurename{List of Figures}
\else
  \newcommand\listfigurename{List of Figures}
\fi
\ifdefined\listtablename
  \renewcommand*\listtablename{List of Tables}
\else
  \newcommand\listtablename{List of Tables}
\fi
\ifdefined\figurename
  \renewcommand*\figurename{Figure}
\else
  \newcommand\figurename{Figure}
\fi
\ifdefined\tablename
  \renewcommand*\tablename{Table}
\else
  \newcommand\tablename{Table}
\fi
}
\@ifpackageloaded{float}{}{\usepackage{float}}
\floatstyle{ruled}
\@ifundefined{c@chapter}{\newfloat{codelisting}{h}{lop}}{\newfloat{codelisting}{h}{lop}[chapter]}
\floatname{codelisting}{Listing}
\newcommand*\listoflistings{\listof{codelisting}{List of Listings}}
\makeatother
\makeatletter
\makeatother
\makeatletter
\@ifpackageloaded{caption}{}{\usepackage{caption}}
\@ifpackageloaded{subcaption}{}{\usepackage{subcaption}}
\makeatother

\usepackage{bookmark}

\IfFileExists{xurl.sty}{\usepackage{xurl}}{} % add URL line breaks if available
\urlstyle{same} % disable monospaced font for URLs
\hypersetup{
  pdftitle={I Programmieren mit Python},
  colorlinks=true,
  linkcolor={blue},
  filecolor={Maroon},
  citecolor={Blue},
  urlcolor={Blue},
  pdfcreator={LaTeX via pandoc}}


\title{I Programmieren mit Python}
\author{}
\date{}

\begin{document}
\maketitle


\subsection{1 Das Setup}\label{das-setup}

\subsubsection{1.1 Die Shell}\label{die-shell}

= Terminal, Eingabeaufforderung

\begin{itemize}
\tightlist
\item
  Beispiele: bash, zsh
\item
  direkte Anweisungen an das Betriebssystem
\item
  mehr Möglichkeiten, mehr Kontrolle
\item
  Programme ausführen
\item
  schneller als Mausbedienung
\item
  grundlegende Terminal-Befehle:
\end{itemize}

\begin{longtable}[]{@{}
  >{\raggedright\arraybackslash}p{(\linewidth - 2\tabcolsep) * \real{0.1233}}
  >{\raggedright\arraybackslash}p{(\linewidth - 2\tabcolsep) * \real{0.8767}}@{}}
\toprule\noalign{}
\begin{minipage}[b]{\linewidth}\raggedright
Befehl
\end{minipage} & \begin{minipage}[b]{\linewidth}\raggedright
Beschreibung
\end{minipage} \\
\midrule\noalign{}
\endhead
\bottomrule\noalign{}
\endlastfoot
\texttt{ls} & list -- Inhalt eines Ordners anzeigen \\
\texttt{cd} & change directory -- Ordner wechseln \\
\texttt{pwd} & print working directory \\
\texttt{mkdir} & make directory -- Ordner erstellen \\
\texttt{cp} & copy -- Dateien oder Ordner kopieren \\
\texttt{mv} & move -- verschieben oder umbenennen \\
\texttt{rm} & remove -- löschen von Dateien oder Ordnern \\
\texttt{nano} & kleiner Editor \\
\texttt{man} & manual -- Hilfe anzeigen \\
\texttt{touch} & Datei anlegen oder Timestamp aktualisieren \\
\end{longtable}

\begin{itemize}
\item
  Programme:

  \begin{itemize}
  \tightlist
  \item
    top: Übersicht über Prozesse
  \item
    vim: Texteditor
  \item
    python3: Python-Interpretor
  \end{itemize}
\item
  Beispiel:
\end{itemize}

\begin{Shaded}
\begin{Highlighting}[numbers=left,,]
\NormalTok{cd }\OperatorTok{\textasciitilde{}/}\NormalTok{workspace}
\end{Highlighting}
\end{Shaded}

\begin{verbatim}
/Users/martin/Workspace
\end{verbatim}

Zeige Inhalte des aktuellen Ordners mit \texttt{ls} an.

\begin{Shaded}
\begin{Highlighting}[numbers=left,,]
\NormalTok{ls}
\end{Highlighting}
\end{Shaded}

\begin{verbatim}
Java_Projects/     Lecture_Projects/  text.txt
Jupyter_Notebooks/ SQL_Projects/      vscode setup/
\end{verbatim}

Erweiterte Anzeige mit Rechten, Besizter, Gruppe, Größe, Zeitstempel

\begin{Shaded}
\begin{Highlighting}[numbers=left,,]
\NormalTok{ls }\OperatorTok{{-}}\NormalTok{al}
\end{Highlighting}
\end{Shaded}

\begin{verbatim}
total 24
drwx------@ 12 martin  staff    384 Jul 25 13:06 ./
drwxr-x---+ 71 martin  staff   2272 Jul 25 12:31 ../
-rw-r--r--@  1 martin  staff  10244 Jul 23 09:02 .DS_Store
drwxr-xr-x   2 martin  staff     64 Mar 25  2024 .ipynb_checkpoints/
-rw-r--r--   1 martin  staff      0 Jul  2  2024 .test
drwxr-xr-x   4 martin  staff    128 Apr 21  2024 .vscode/
drwx------@  7 martin  staff    224 Apr 14  2024 Java_Projects/
drwxr-xr-x@ 10 martin  staff    320 Jan 30 14:39 Jupyter_Notebooks/
drwxr-xr-x   7 martin  staff    224 Jul 23 09:01 Lecture_Projects/
drwxr-xr-x   9 martin  staff    288 Jan 26 12:51 SQL_Projects/
-rw-r--r--@  1 martin  staff      0 Jul 25 13:06 text.txt
drwxr-xr-x@  3 martin  staff     96 Feb 14 12:46 vscode setup/
\end{verbatim}

Name des aktuellen Ornders

\begin{Shaded}
\begin{Highlighting}[numbers=left,,]
\NormalTok{pwd}
\end{Highlighting}
\end{Shaded}

\begin{verbatim}
'/Users/martin/Workspace'
\end{verbatim}

Neue Verzeichnis mit Namen ``mydata'' erstellen.

\begin{Shaded}
\begin{Highlighting}[numbers=left,,]
\NormalTok{mkdir mydata}
\end{Highlighting}
\end{Shaded}

Überprüfen, ob der neue Ornder auch angelegt worden ist.

\begin{Shaded}
\begin{Highlighting}[numbers=left,,]
\NormalTok{ls }\OperatorTok{{-}}\NormalTok{al}
\end{Highlighting}
\end{Shaded}

\begin{verbatim}
total 24
drwx------@ 13 martin  staff    416 Jul 25 13:09 ./
drwxr-x---+ 71 martin  staff   2272 Jul 25 12:31 ../
-rw-r--r--@  1 martin  staff  10244 Jul 23 09:02 .DS_Store
drwxr-xr-x   2 martin  staff     64 Mar 25  2024 .ipynb_checkpoints/
-rw-r--r--   1 martin  staff      0 Jul  2  2024 .test
drwxr-xr-x   4 martin  staff    128 Apr 21  2024 .vscode/
drwx------@  7 martin  staff    224 Apr 14  2024 Java_Projects/
drwxr-xr-x@ 10 martin  staff    320 Jan 30 14:39 Jupyter_Notebooks/
drwxr-xr-x   7 martin  staff    224 Jul 23 09:01 Lecture_Projects/
drwxr-xr-x@  2 martin  staff     64 Jul 25 13:09 mydata/
drwxr-xr-x   9 martin  staff    288 Jan 26 12:51 SQL_Projects/
-rw-r--r--@  1 martin  staff      0 Jul 25 13:06 text.txt
drwxr-xr-x@  3 martin  staff     96 Feb 14 12:46 vscode setup/
\end{verbatim}

In den neune Ornder ``mydata'' wechseln.

\begin{Shaded}
\begin{Highlighting}[numbers=left,,]
\NormalTok{cd mydata}
\end{Highlighting}
\end{Shaded}

\begin{verbatim}
/Users/martin/Workspace/mydata
\end{verbatim}

Inhalt des Ordners ``mydata'' anzeigen. Er ist leer.

\begin{Shaded}
\begin{Highlighting}[numbers=left,,]
\NormalTok{ls}
\end{Highlighting}
\end{Shaded}

Neue Datei namens text.txt. erstellen.

\begin{Shaded}
\begin{Highlighting}[numbers=left,,]
\OperatorTok{!}\NormalTok{touch text.txt}
\end{Highlighting}
\end{Shaded}

Überprüfen, ob die neue Datei auch vorhanden ist.

\begin{Shaded}
\begin{Highlighting}[numbers=left,,]
\NormalTok{ls }
\end{Highlighting}
\end{Shaded}

\begin{verbatim}
text.txt
\end{verbatim}

Aus dem Ornder ``mydata'' in den übergeordneten Ordner mit einer
relativen (zum aktuellen Ordner) Pfad wechseln. Der übergeordente Ordner
wird mit .. adressiert.

\begin{Shaded}
\begin{Highlighting}[numbers=left,,]
\NormalTok{cd ..}
\end{Highlighting}
\end{Shaded}

\begin{verbatim}
/Users/martin/Workspace
\end{verbatim}

Weiteres Beispiel eines relativen (relativ zum aktuellen Ornder) Pfades.

\begin{Shaded}
\begin{Highlighting}[numbers=left,,]
\NormalTok{cd mydata}\OperatorTok{/}\NormalTok{..}\OperatorTok{/}\NormalTok{..}
\end{Highlighting}
\end{Shaded}

\begin{verbatim}
/Users/martin
\end{verbatim}

Aus dem aktuellen Ornder in den Unter- Unterordner ``mydata'' wechseln.

\begin{Shaded}
\begin{Highlighting}[numbers=left,,]
\NormalTok{cd Workspace}\OperatorTok{/}\NormalTok{mydata}\OperatorTok{/}
\end{Highlighting}
\end{Shaded}

\begin{verbatim}
/Users/martin/Workspace/mydata
\end{verbatim}

Der akuteller Ordner ist nun ``mydata''. In ihm befindet sich die Datei
``text.txt''. Diese soll mit der Anweisung \texttt{cp} in den
übergordneten Ordner kopiert werden.

allgemeine Syntax: \texttt{cp\ source-file\ target-file}

\begin{Shaded}
\begin{Highlighting}[numbers=left,,]
\NormalTok{cp text.txt .. }
\end{Highlighting}
\end{Shaded}

Alternative: Man wechselt in den Zielornder und adressiert diesen mit .

Syntax: \texttt{cp\ source-file\ .}

Wir wechseln daher in den übergeordenten Ordner:

\begin{Shaded}
\begin{Highlighting}[numbers=left,,]
\NormalTok{cd ..}
\end{Highlighting}
\end{Shaded}

\begin{verbatim}
/Users/martin/Workspace
\end{verbatim}

und kopieren dann die Datei ``text.txt'' in diesen hinein.

\begin{Shaded}
\begin{Highlighting}[numbers=left,,]
\NormalTok{cp mydata}\OperatorTok{/}\NormalTok{text.txt .}
\end{Highlighting}
\end{Shaded}

Überprüfung des Ergebnisses mit \texttt{ls}

\begin{Shaded}
\begin{Highlighting}[numbers=left,,]
\NormalTok{ls}
\end{Highlighting}
\end{Shaded}

\begin{verbatim}
Java_Projects/     Lecture_Projects/  SQL_Projects/      vscode setup/
Jupyter_Notebooks/ mydata/            text.txt
\end{verbatim}

Nun soll die doppelte Datei ``text.txt'' im Unterordner ``mydata''
gelöscht werden:

\begin{Shaded}
\begin{Highlighting}[numbers=left,,]
\NormalTok{rm mydata}\OperatorTok{/}\NormalTok{text.txt}
\end{Highlighting}
\end{Shaded}

\begin{Shaded}
\begin{Highlighting}[numbers=left,,]
\NormalTok{ls mydata}
\end{Highlighting}
\end{Shaded}

Welche Texte sind in der Datei ``text.txt'' gespeichert? Um in die Datei
zu schauen, muss man einen Editor, am einfachsten einen Editor des
Terminals verwenden.

Commandline Editoren:

\begin{itemize}
\tightlist
\item
  vim: hohe Lernkurve, mächtig, wird von vielen Programmierern und
  Systemadministratoren verwendet.
\item
  Aufruf: \texttt{vim\ text.txt}
\item
  nano: intuitiver zu bedienen, nicht so mächtig.
\item
  Aufruf: \texttt{nano\ text.txt}
\end{itemize}

Die Datei ``text.txt'' und den Ordner ``mydata'' benötigen wir nun nicht
mehr und löschen diese. Anstatt die Datei ``text.txt'' mit dem Befehl
\texttt{rm\ text.txt"} und den Ordner mit dem Befehl
\texttt{rmdir\ mydata}getrennt zu löschen, werden wir den Ordner samt
und Datei einzeln mit dem Befehl \texttt{rm}zu löschen, verwenden wir
den Befehl \texttt{rmdir}mit dem Flag -R (R steth für rekursiv), der
dafür sorgt, dass der Ordner mitsamt Inhalt gelöscht wird. Damit das
funktioniert, darf der aktuelle Ordner nicht der zu löschende Ordner
oder ein Unterordner sein.

\begin{Shaded}
\begin{Highlighting}[numbers=left,,]
\NormalTok{rm }\OperatorTok{{-}}\NormalTok{R mydata}
\end{Highlighting}
\end{Shaded}

Informationen zu einem bestimmten Terminal-Befehl erhält man mit der
\texttt{man}-Anweisung:

\begin{Shaded}
\begin{Highlighting}[numbers=left,,]
\NormalTok{man rmdir}
\end{Highlighting}
\end{Shaded}

\begin{itemize}
\tightlist
\item
  \href{1_Programming_Introduction/1_Python_Programming/1-1_Setup/1-1_Setup.ipynb}{CheatSheat
  zum Terminal (pdf)}
\end{itemize}

\subsubsection{1.2 Microsoft Visual Studio Code (VS
Code)}\label{microsoft-visual-studio-code-vs-code}

Visual Studio Code (VS Code) ist mehr als ein Texteditor, aber schlanker
als eine vollwertige IDE. Es bietet komfortables Programmieren mit
Features wie Autovervollständigung und Syntaxhervorhebung. Ein
integriertes Terminal erleichtert den Workflow. Durch Extensions lässt
sich VS Code flexibel erweitern und anpassen.

\href{1_Programming_Introduction/1_Python_Programming/1-1_Setup/1-1_CheatSheet_VSCode.pdf}{CheatSheet
zu VS Code}

\subsubsection{1.3 Jupyter Notebook}\label{jupyter-notebook}

\begin{itemize}
\tightlist
\item
  quelloffene und kostenlose WebAnwendung
\item
  Integriert

  \begin{itemize}
  \tightlist
  \item
    Markdown Umgebung
  \item
    (Python) Umgebung
  \end{itemize}
\item
  Verwendung von Data Scientists
\item
  Austausch von Daten
\item
  bis zu 40 Programmiersprachen werden unterstützt
\item
  Dokumente gut lesbar
\end{itemize}

\subsubsection{1.4 Markdown Language}\label{markdown-language}

Mardown ist eine einfache Auszeichnungssprache. Sie ist leicht zu
erlernen. Die wichtigsten Tags sind im Folgenden zusammengefasst.

\begin{longtable}[]{@{}ll@{}}
\toprule\noalign{}
Ergebnis & Markdown-Code \\
\midrule\noalign{}
\endhead
\bottomrule\noalign{}
\endlastfoot
Überschrift 1 & \texttt{\#\ Überschrift\ 1} \\
Überschrift 2 & \texttt{\#\#\ Überschrift\ 2} \\
Überschrift 3 & \texttt{\#\#\#\ Überschrift\ 3} \\
Überschrift 4 & \texttt{\#\#\#\#\ Überschrift\ 4} \\
Überschrift 5 & \texttt{\#\#\#\#\#\ Überschrift\ 5} \\
\textbf{fetter Text} & \texttt{**fetterText**} \\
\emph{kursiver Text} & \texttt{*kursiver\ Text*} \\
• Element & \texttt{*\ Element} \\
\textgreater{} Zitat & \texttt{\textgreater{}\ Zitat} \\
\texttt{InlineCode} &
\texttt{\textasciigrave{}InlineCode\textasciigrave{}} \\
\texttt{Codeblock} &
\texttt{\textasciigrave{}\textasciigrave{}\textasciigrave{}Codeblock\textasciigrave{}\textasciigrave{}\textasciigrave{}} \\
--- horizontale Linie & \texttt{-\/-\/-\/-} \\
* escaped Stern & \texttt{\textbackslash{}*\ escaped\ Stern} \\
- Element - Unterelement &
\texttt{-\ Element\textbackslash{}n\ \ -\ Unterelement} \\
\end{longtable}

\href{1_Programming_Introduction/1_Python_Programming/1-1_Setup/1-1_Markdown_CheatSheet.pdf}{CheatSheat
zu Markdown (pdf)}

\subsubsection{1.5 LaTeX}\label{latex}

LaTeX ist ein Textsatzsystem, das besonders für wissenschaftliche
Arbeiten und Dokumente verwendet wird. Es eignet sich hervorragend zum
Schreiben von Texten mit Formeln, Tabellen und Literaturverzeichnissen.
Anders als in Word wird der Text in LaTeX mit Befehlen und Codes
geschrieben.

Innerhalb der Markdown-Umgebung kann LaTeX für mathematische Formeln
verwendet werden. Hier sind einige grundlegende LaTeX-Befehle:

\begin{longtable}[]{@{}
  >{\raggedright\arraybackslash}p{(\linewidth - 2\tabcolsep) * \real{0.5106}}
  >{\raggedright\arraybackslash}p{(\linewidth - 2\tabcolsep) * \real{0.4894}}@{}}
\toprule\noalign{}
\begin{minipage}[b]{\linewidth}\raggedright
Beschreibung
\end{minipage} & \begin{minipage}[b]{\linewidth}\raggedright
LaTeX-Befehl
\end{minipage} \\
\midrule\noalign{}
\endhead
\bottomrule\noalign{}
\endlastfoot
Inline-Formeln & \texttt{\$\textbackslash{}text\{Formel\}\ \$} \\
Block-Formeln & \texttt{\$\$\ \textbackslash{}text\{Formel\}\ \$\$} \\
\(\alpha\) & \texttt{\textbackslash{}alpha} \\
\(\frac{z}{n}\) & \texttt{\textbackslash{}frac\{z\}\{n\}} \\
\(\sqrt{a}\) & \texttt{\textbackslash{}sqrt\{a\}} \\
\(\sum_{i=1}^{n} i\) &
\texttt{\textbackslash{}sum\_\{i=1\}\^{}\{n\}\ i} \\
\(\int_{a}^{b} f(x) \, dx\) &
\texttt{\textbackslash{}int\_\{a\}\^{}\{b\}\ f(x)\ \textbackslash{},\ dx} \\
\(\vec{v}\) & \texttt{\textbackslash{}vec\{v\}} \\
\(\begin{pmatrix} a & b \\ c & d \end{pmatrix}\) &
\texttt{\textbackslash{}begin\{pmatrix\}\ a\ \&\ b\ \textbackslash{}\textbackslash{}\ c\ \&\ d\ \textbackslash{}end\{pmatrix\}} \\
\(\begin{aligned} a &= b \\ c &= d \end{aligned}\) &
\texttt{\textbackslash{}begin\{align\}\ a\ \&=\ b\ \textbackslash{}\textbackslash{}\ c\ \&=\ d\ \textbackslash{}end\{align\}} \\
\end{longtable}

\href{1_Programming_Introduction/1_Python_Programming/1-1_Setup/1-1_Markdown_LaTeX_CheatSheet.pdf}{CheatSheat
zu LaTeX (pdf)}




\end{document}
