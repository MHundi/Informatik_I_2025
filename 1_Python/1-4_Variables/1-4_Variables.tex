% Options for packages loaded elsewhere
\PassOptionsToPackage{unicode}{hyperref}
\PassOptionsToPackage{hyphens}{url}
\PassOptionsToPackage{dvipsnames,svgnames,x11names}{xcolor}
%
\documentclass[
  11pt,
  a4paper,
  DIV=11,
  numbers=noendperiod]{scrartcl}

\usepackage{amsmath,amssymb}
\usepackage{iftex}
\ifPDFTeX
  \usepackage[T1]{fontenc}
  \usepackage[utf8]{inputenc}
  \usepackage{textcomp} % provide euro and other symbols
\else % if luatex or xetex
  \usepackage{unicode-math}
  \defaultfontfeatures{Scale=MatchLowercase}
  \defaultfontfeatures[\rmfamily]{Ligatures=TeX,Scale=1}
\fi
\usepackage{lmodern}
\ifPDFTeX\else  
    % xetex/luatex font selection
    \setmainfont[]{Avenir Next}
\fi
% Use upquote if available, for straight quotes in verbatim environments
\IfFileExists{upquote.sty}{\usepackage{upquote}}{}
\IfFileExists{microtype.sty}{% use microtype if available
  \usepackage[]{microtype}
  \UseMicrotypeSet[protrusion]{basicmath} % disable protrusion for tt fonts
}{}
\makeatletter
\@ifundefined{KOMAClassName}{% if non-KOMA class
  \IfFileExists{parskip.sty}{%
    \usepackage{parskip}
  }{% else
    \setlength{\parindent}{0pt}
    \setlength{\parskip}{6pt plus 2pt minus 1pt}}
}{% if KOMA class
  \KOMAoptions{parskip=half}}
\makeatother
\usepackage{xcolor}
\usepackage[top=3cm, bottom=2cm, left=4cm, right=2cm]{geometry}
\setlength{\emergencystretch}{3em} % prevent overfull lines
\setcounter{secnumdepth}{-\maxdimen} % remove section numbering
% Make \paragraph and \subparagraph free-standing
\makeatletter
\ifx\paragraph\undefined\else
  \let\oldparagraph\paragraph
  \renewcommand{\paragraph}{
    \@ifstar
      \xxxParagraphStar
      \xxxParagraphNoStar
  }
  \newcommand{\xxxParagraphStar}[1]{\oldparagraph*{#1}\mbox{}}
  \newcommand{\xxxParagraphNoStar}[1]{\oldparagraph{#1}\mbox{}}
\fi
\ifx\subparagraph\undefined\else
  \let\oldsubparagraph\subparagraph
  \renewcommand{\subparagraph}{
    \@ifstar
      \xxxSubParagraphStar
      \xxxSubParagraphNoStar
  }
  \newcommand{\xxxSubParagraphStar}[1]{\oldsubparagraph*{#1}\mbox{}}
  \newcommand{\xxxSubParagraphNoStar}[1]{\oldsubparagraph{#1}\mbox{}}
\fi
\makeatother

\usepackage{color}
\usepackage{fancyvrb}
\newcommand{\VerbBar}{|}
\newcommand{\VERB}{\Verb[commandchars=\\\{\}]}
\DefineVerbatimEnvironment{Highlighting}{Verbatim}{commandchars=\\\{\}}
% Add ',fontsize=\small' for more characters per line
\usepackage{framed}
\definecolor{shadecolor}{RGB}{241,243,245}
\newenvironment{Shaded}{\begin{snugshade}}{\end{snugshade}}
\newcommand{\AlertTok}[1]{\textcolor[rgb]{0.68,0.00,0.00}{#1}}
\newcommand{\AnnotationTok}[1]{\textcolor[rgb]{0.37,0.37,0.37}{#1}}
\newcommand{\AttributeTok}[1]{\textcolor[rgb]{0.40,0.45,0.13}{#1}}
\newcommand{\BaseNTok}[1]{\textcolor[rgb]{0.68,0.00,0.00}{#1}}
\newcommand{\BuiltInTok}[1]{\textcolor[rgb]{0.00,0.23,0.31}{#1}}
\newcommand{\CharTok}[1]{\textcolor[rgb]{0.13,0.47,0.30}{#1}}
\newcommand{\CommentTok}[1]{\textcolor[rgb]{0.37,0.37,0.37}{#1}}
\newcommand{\CommentVarTok}[1]{\textcolor[rgb]{0.37,0.37,0.37}{\textit{#1}}}
\newcommand{\ConstantTok}[1]{\textcolor[rgb]{0.56,0.35,0.01}{#1}}
\newcommand{\ControlFlowTok}[1]{\textcolor[rgb]{0.00,0.23,0.31}{\textbf{#1}}}
\newcommand{\DataTypeTok}[1]{\textcolor[rgb]{0.68,0.00,0.00}{#1}}
\newcommand{\DecValTok}[1]{\textcolor[rgb]{0.68,0.00,0.00}{#1}}
\newcommand{\DocumentationTok}[1]{\textcolor[rgb]{0.37,0.37,0.37}{\textit{#1}}}
\newcommand{\ErrorTok}[1]{\textcolor[rgb]{0.68,0.00,0.00}{#1}}
\newcommand{\ExtensionTok}[1]{\textcolor[rgb]{0.00,0.23,0.31}{#1}}
\newcommand{\FloatTok}[1]{\textcolor[rgb]{0.68,0.00,0.00}{#1}}
\newcommand{\FunctionTok}[1]{\textcolor[rgb]{0.28,0.35,0.67}{#1}}
\newcommand{\ImportTok}[1]{\textcolor[rgb]{0.00,0.46,0.62}{#1}}
\newcommand{\InformationTok}[1]{\textcolor[rgb]{0.37,0.37,0.37}{#1}}
\newcommand{\KeywordTok}[1]{\textcolor[rgb]{0.00,0.23,0.31}{\textbf{#1}}}
\newcommand{\NormalTok}[1]{\textcolor[rgb]{0.00,0.23,0.31}{#1}}
\newcommand{\OperatorTok}[1]{\textcolor[rgb]{0.37,0.37,0.37}{#1}}
\newcommand{\OtherTok}[1]{\textcolor[rgb]{0.00,0.23,0.31}{#1}}
\newcommand{\PreprocessorTok}[1]{\textcolor[rgb]{0.68,0.00,0.00}{#1}}
\newcommand{\RegionMarkerTok}[1]{\textcolor[rgb]{0.00,0.23,0.31}{#1}}
\newcommand{\SpecialCharTok}[1]{\textcolor[rgb]{0.37,0.37,0.37}{#1}}
\newcommand{\SpecialStringTok}[1]{\textcolor[rgb]{0.13,0.47,0.30}{#1}}
\newcommand{\StringTok}[1]{\textcolor[rgb]{0.13,0.47,0.30}{#1}}
\newcommand{\VariableTok}[1]{\textcolor[rgb]{0.07,0.07,0.07}{#1}}
\newcommand{\VerbatimStringTok}[1]{\textcolor[rgb]{0.13,0.47,0.30}{#1}}
\newcommand{\WarningTok}[1]{\textcolor[rgb]{0.37,0.37,0.37}{\textit{#1}}}

\providecommand{\tightlist}{%
  \setlength{\itemsep}{0pt}\setlength{\parskip}{0pt}}\usepackage{longtable,booktabs,array}
\usepackage{calc} % for calculating minipage widths
% Correct order of tables after \paragraph or \subparagraph
\usepackage{etoolbox}
\makeatletter
\patchcmd\longtable{\par}{\if@noskipsec\mbox{}\fi\par}{}{}
\makeatother
% Allow footnotes in longtable head/foot
\IfFileExists{footnotehyper.sty}{\usepackage{footnotehyper}}{\usepackage{footnote}}
\makesavenoteenv{longtable}
\usepackage{graphicx}
\makeatletter
\newsavebox\pandoc@box
\newcommand*\pandocbounded[1]{% scales image to fit in text height/width
  \sbox\pandoc@box{#1}%
  \Gscale@div\@tempa{\textheight}{\dimexpr\ht\pandoc@box+\dp\pandoc@box\relax}%
  \Gscale@div\@tempb{\linewidth}{\wd\pandoc@box}%
  \ifdim\@tempb\p@<\@tempa\p@\let\@tempa\@tempb\fi% select the smaller of both
  \ifdim\@tempa\p@<\p@\scalebox{\@tempa}{\usebox\pandoc@box}%
  \else\usebox{\pandoc@box}%
  \fi%
}
% Set default figure placement to htbp
\def\fps@figure{htbp}
\makeatother

\usepackage[document]{ragged2e}
\KOMAoption{captions}{tableheading}
\makeatletter
\@ifpackageloaded{tcolorbox}{}{\usepackage[skins,breakable]{tcolorbox}}
\@ifpackageloaded{fontawesome5}{}{\usepackage{fontawesome5}}
\definecolor{quarto-callout-color}{HTML}{909090}
\definecolor{quarto-callout-note-color}{HTML}{0758E5}
\definecolor{quarto-callout-important-color}{HTML}{CC1914}
\definecolor{quarto-callout-warning-color}{HTML}{EB9113}
\definecolor{quarto-callout-tip-color}{HTML}{00A047}
\definecolor{quarto-callout-caution-color}{HTML}{FC5300}
\definecolor{quarto-callout-color-frame}{HTML}{acacac}
\definecolor{quarto-callout-note-color-frame}{HTML}{4582ec}
\definecolor{quarto-callout-important-color-frame}{HTML}{d9534f}
\definecolor{quarto-callout-warning-color-frame}{HTML}{f0ad4e}
\definecolor{quarto-callout-tip-color-frame}{HTML}{02b875}
\definecolor{quarto-callout-caution-color-frame}{HTML}{fd7e14}
\makeatother
\makeatletter
\@ifpackageloaded{caption}{}{\usepackage{caption}}
\AtBeginDocument{%
\ifdefined\contentsname
  \renewcommand*\contentsname{Table of contents}
\else
  \newcommand\contentsname{Table of contents}
\fi
\ifdefined\listfigurename
  \renewcommand*\listfigurename{List of Figures}
\else
  \newcommand\listfigurename{List of Figures}
\fi
\ifdefined\listtablename
  \renewcommand*\listtablename{List of Tables}
\else
  \newcommand\listtablename{List of Tables}
\fi
\ifdefined\figurename
  \renewcommand*\figurename{Figure}
\else
  \newcommand\figurename{Figure}
\fi
\ifdefined\tablename
  \renewcommand*\tablename{Table}
\else
  \newcommand\tablename{Table}
\fi
}
\@ifpackageloaded{float}{}{\usepackage{float}}
\floatstyle{ruled}
\@ifundefined{c@chapter}{\newfloat{codelisting}{h}{lop}}{\newfloat{codelisting}{h}{lop}[chapter]}
\floatname{codelisting}{Listing}
\newcommand*\listoflistings{\listof{codelisting}{List of Listings}}
\makeatother
\makeatletter
\makeatother
\makeatletter
\@ifpackageloaded{caption}{}{\usepackage{caption}}
\@ifpackageloaded{subcaption}{}{\usepackage{subcaption}}
\makeatother
\makeatletter
\definecolor{QuartoInternalColor3}{rgb}{0.00,0.45,0.15}
\definecolor{QuartoInternalColor4}{rgb}{0.00,0.40,0.00}
\definecolor{QuartoInternalColor2}{rgb}{0,0,0}
\definecolor{QuartoInternalColor1}{rgb}{0.70,0.17,0.19}
\definecolor{QuartoInternalColor6}{rgb}{0.70,0.49,0.07}
\definecolor{QuartoInternalColor5}{rgb}{0.15,0.56,0.56}
\makeatother

\usepackage{bookmark}

\IfFileExists{xurl.sty}{\usepackage{xurl}}{} % add URL line breaks if available
\urlstyle{same} % disable monospaced font for URLs
\hypersetup{
  pdftitle={4. Variablen},
  colorlinks=true,
  linkcolor={blue},
  filecolor={Maroon},
  citecolor={Blue},
  urlcolor={Blue},
  pdfcreator={LaTeX via pandoc}}


\title{4. Variablen}
\author{}
\date{}

\begin{document}
\maketitle


Variablen sind ein fundamentales Konzept in Programmiersprachen, da sie
es ermöglichen, Daten temporär im Speicher zu speichern und später
wieder abzurufen. Ohne Variablen könnten Programme keine Eingaben
verarbeiten, Zwischenergebnisse speichern oder auf veränderte Daten
reagieren. Sie fungieren als Container für Werte, die sich während der
Programmausführung ändern können, was die Grundlage für dynamische und
interaktive Programme bildet. Variablen machen Programme flexibel und
wiederverwendbar, da derselbe Code mit unterschiedlichen Daten arbeiten
kann. Darüber hinaus verbessern sie die Lesbarkeit und Wartbarkeit von
Code, indem sie aussagekräftige Namen für Daten bereitstellen.

In Python sind folgende Datentypen für Variablen möglich:

\begin{itemize}
\tightlist
\item
  \texttt{int} für Ganzzahlen
\item
  \texttt{float} für Fließkommazahlen
\item
  \texttt{string} für Zeichenketten
\item
  \texttt{boolean} für Wahrheitswerte
\item
  \texttt{list} für Listen
\item
  \texttt{tuple} für Tupel
\end{itemize}

\paragraph{Beispiel}\label{beispiel}

\begin{Shaded}
\begin{Highlighting}[numbers=left,,]
\NormalTok{note }\OperatorTok{=} \DecValTok{123}
\BuiltInTok{print}\NormalTok{(note)}
\end{Highlighting}
\end{Shaded}

\begin{verbatim}
123
\end{verbatim}

\paragraph{Syntax}\label{syntax}

\begin{tcolorbox}[enhanced jigsaw, opacitybacktitle=0.6, left=2mm, coltitle=black, colframe=quarto-callout-note-color-frame, breakable, colbacktitle=quarto-callout-note-color!10!white, arc=.35mm, rightrule=.15mm, bottomrule=.15mm, bottomtitle=1mm, colback=white, toptitle=1mm, titlerule=0mm, leftrule=.75mm, toprule=.15mm, title=\textcolor{quarto-callout-note-color}{\faInfo}\hspace{0.5em}{Syntax}, opacityback=0]

\texttt{Bezeichner\ =\ Ausdruck}

\end{tcolorbox}

\paragraph{Bemerkungen}\label{bemerkungen}

\begin{itemize}
\tightlist
\item
  Python ist eine dynamische Sprache, d.h. Variablen müssen nicht
  deklariert werden. Der Typ einer Variablen wird durch den Wert
  bestimmt, der ihr zugewiesen wird.
\item
  Pro Zeile nur eine Anweisung!
\item
  linke Seite: Bezeichner = Variablennamen = Identifier
\item
  rechte Seite: Ausdruck oder ein Wert
\item
  Gleichheitszeichen ist der Zuweisungsoperator
\item
  Variablen sind erst nach einer Anweisung verwendbar.
\item
  Erst wird die rechte Seite ausgewertet und dann an die Variable
  zugewiesen.
\end{itemize}

\begin{tcolorbox}[enhanced jigsaw, opacitybacktitle=0.6, left=2mm, coltitle=black, colframe=quarto-callout-warning-color-frame, breakable, colbacktitle=quarto-callout-warning-color!10!white, arc=.35mm, rightrule=.15mm, bottomrule=.15mm, bottomtitle=1mm, colback=white, toptitle=1mm, titlerule=0mm, leftrule=.75mm, toprule=.15mm, title=\textcolor{quarto-callout-warning-color}{\faExclamationTriangle}\hspace{0.5em}{Achtung}, opacityback=0]

Das Gleichheitszeichen ist der Zuweisungsoperator und nicht das
mathematische Gleichheitszeichen (Vergleichsoperator).

\end{tcolorbox}

\paragraph{Regeln für die Wahl des
Bezeichners}\label{regeln-fuxfcr-die-wahl-des-bezeichners}

\begin{itemize}
\tightlist
\item
  Buchstaben, Unterstriche und Ziffern
\item
  erste Zeichen keine Ziffer
\item
  keine Leerzeichen
\item
  keine Python Schlüsselwörter (if, else, return, class, or,\ldots)
\item
  Python ist case-sensitive, d.h. Groß- und Kleinschreibung
  unterscheiden sich.
\end{itemize}

\paragraph{Ausdrücke}\label{ausdruxfccke}

Ausdrücke können sein: - Operatoren - Literalen - Variablen

Die Auswertung eines Ausdrucks liefert einen Wert oder bricht mit
Fehlermeldung ab. Die Auswertung bei arithmetischen Ausdrücken läuft
nach folgenden Regeln ab:

\begin{enumerate}
\def\labelenumi{\arabic{enumi}.}
\tightlist
\item
  Klammern zuerst
\item
  Potenzen
\item
  Multiplikation / Division
\item
  Addition / Subtraktion
\item
  ansonsten von links nach rechts.
\end{enumerate}

\paragraph{Beispiele:}\label{beispiele}

Zuweisung von Werten an Variablen

\begin{Shaded}
\begin{Highlighting}[numbers=left,,]
\NormalTok{note }\OperatorTok{=} \DecValTok{4}
\end{Highlighting}
\end{Shaded}

Ausgabe der Variablen

\begin{Shaded}
\begin{Highlighting}[numbers=left,,]
\BuiltInTok{print}\NormalTok{(note)}
\end{Highlighting}
\end{Shaded}

\begin{verbatim}
4
\end{verbatim}

Mit Variablen rechnen

\begin{Shaded}
\begin{Highlighting}[numbers=left,,]
\BuiltInTok{print}\NormalTok{(}\DecValTok{2}\OperatorTok{*}\DecValTok{4}\OperatorTok{**}\NormalTok{note)}
\end{Highlighting}
\end{Shaded}

\begin{verbatim}
512
\end{verbatim}

Dynamische Typisierung

\begin{Shaded}
\begin{Highlighting}[numbers=left,,]
\NormalTok{durchschnitt }\OperatorTok{=} \FloatTok{12.3}
\BuiltInTok{print}\NormalTok{(durchschnitt)}
\end{Highlighting}
\end{Shaded}

\begin{verbatim}
12.3
\end{verbatim}

Rechnen mit unterschiedlichen Zahldatentypen

\begin{Shaded}
\begin{Highlighting}[numbers=left,,]
\BuiltInTok{print}\NormalTok{(durchschnitt }\OperatorTok{*}\NormalTok{ note)}
\end{Highlighting}
\end{Shaded}

\begin{verbatim}
49.2
\end{verbatim}

Fehler bei nicht definierten Variablen

\begin{Shaded}
\begin{Highlighting}[numbers=left,,]
\BuiltInTok{print}\NormalTok{(goody)}
\end{Highlighting}
\end{Shaded}

\begin{Highlighting}
\textcolor{black}{NameError: name 'goody' is not defined}
\textcolor{black}{}\textcolor{QuartoInternalColor1}{---------------------------------------------------------------------------}\textcolor{QuartoInternalColor2}{}
\textcolor{QuartoInternalColor2}{}\textcolor{QuartoInternalColor1}{NameError}\textcolor{QuartoInternalColor2}{                                 Traceback (most recent call last)}
\textcolor{QuartoInternalColor2}{Cell }\textcolor{QuartoInternalColor3}{In[7], line 1}\textcolor{QuartoInternalColor2}{}
\textcolor{QuartoInternalColor2}{}\textcolor{QuartoInternalColor3}{----> 1}\textcolor{QuartoInternalColor2}{ }\textcolor{QuartoInternalColor4}{print}\textcolor{QuartoInternalColor2}{(}\textcolor{QuartoInternalColor2}{goody}\textcolor{QuartoInternalColor2}{)}
\textcolor{QuartoInternalColor2}{}\textcolor{QuartoInternalColor1}{NameError}\textcolor{QuartoInternalColor2}{: name 'goody' is not defined}
\end{Highlighting}

Strings sind Zeichenketten, die in Anführungszeichen stehen. Sie können
mit Variablen verknüpft werden.

\begin{Shaded}
\begin{Highlighting}[numbers=left,,]
\BuiltInTok{print}\NormalTok{(}\StringTok{"ham"}\NormalTok{)}
\end{Highlighting}
\end{Shaded}

\begin{verbatim}
ham
\end{verbatim}

Analog geht das auch mit einfachen Anführungszeichen

\begin{Shaded}
\begin{Highlighting}[numbers=left,,]
\BuiltInTok{print}\NormalTok{(}\StringTok{\textquotesingle{}egg\textquotesingle{}}\NormalTok{)}
\end{Highlighting}
\end{Shaded}

\begin{tcolorbox}[enhanced jigsaw, opacitybacktitle=0.6, left=2mm, coltitle=black, colframe=quarto-callout-tip-color-frame, breakable, colbacktitle=quarto-callout-tip-color!10!white, arc=.35mm, rightrule=.15mm, bottomrule=.15mm, bottomtitle=1mm, colback=white, toptitle=1mm, titlerule=0mm, leftrule=.75mm, toprule=.15mm, title=\textcolor{quarto-callout-tip-color}{\faLightbulb}\hspace{0.5em}{Info}, opacityback=0]

Der PEP 8 (Python Style Guide) empfiehlt, Strings in einfachen
Anführungszeichen zu schreiben.

\end{tcolorbox}

Der +-Operator wird bei Strings für die Verkettung von Strings
(Konkatenation)verwendet.

\begin{Shaded}
\begin{Highlighting}[numbers=left,,]
\BuiltInTok{print}\NormalTok{(}\StringTok{\textquotesingle{}ham\textquotesingle{}}\OperatorTok{+}\StringTok{\textquotesingle{}egg\textquotesingle{}}\NormalTok{)}
\end{Highlighting}
\end{Shaded}

\begin{verbatim}
hamegg
\end{verbatim}

Stings lassen sich in Python auch mit ganzen Zahlen über den *-Operator
verknüpfen, was zu einer Wiederholung des Strings führt.

\begin{Shaded}
\begin{Highlighting}[numbers=left,,]
\BuiltInTok{print}\NormalTok{(}\StringTok{\textquotesingle{}ham\textquotesingle{}}\OperatorTok{*}\DecValTok{3}\NormalTok{)}
\end{Highlighting}
\end{Shaded}

\begin{verbatim}
hamhamham
\end{verbatim}

\begin{Shaded}
\begin{Highlighting}[numbers=left,,]
\BuiltInTok{print}\NormalTok{(}\StringTok{\textquotesingle{}ham\textquotesingle{}}\OperatorTok{*}\FloatTok{1.5}\NormalTok{)}
\end{Highlighting}
\end{Shaded}

\begin{Highlighting}
\textcolor{black}{TypeError: can't multiply sequence by non-int of type 'float'}
\textcolor{black}{}\textcolor{QuartoInternalColor1}{---------------------------------------------------------------------------}\textcolor{QuartoInternalColor2}{}
\textcolor{QuartoInternalColor2}{}\textcolor{QuartoInternalColor1}{TypeError}\textcolor{QuartoInternalColor2}{                                 Traceback (most recent call last)}
\textcolor{QuartoInternalColor2}{}\textcolor{QuartoInternalColor5}{Cell}\textcolor{QuartoInternalColor2}{}\textcolor{QuartoInternalColor5}{ }\textcolor{QuartoInternalColor2}{}\textcolor{QuartoInternalColor3}{In[21]}\textcolor{QuartoInternalColor2}{}\textcolor{QuartoInternalColor3}{, line 1}\textcolor{QuartoInternalColor2}{}
\textcolor{QuartoInternalColor2}{}\textcolor{QuartoInternalColor3}{----> }\textcolor{QuartoInternalColor2}{}\textcolor{QuartoInternalColor3}{1}\textcolor{QuartoInternalColor2}{ }\textcolor{QuartoInternalColor4}{print}\textcolor{QuartoInternalColor2}{(}\textcolor{QuartoInternalColor2}{'}\textcolor{QuartoInternalColor2}{}\textcolor{QuartoInternalColor2}{ham}\textcolor{QuartoInternalColor2}{}\textcolor{QuartoInternalColor2}{'}\textcolor{QuartoInternalColor2}{}\textcolor{QuartoInternalColor2}{*}\textcolor{QuartoInternalColor2}{}\textcolor{QuartoInternalColor2}{1.5}\textcolor{QuartoInternalColor2}{)}
\textcolor{QuartoInternalColor2}{}\textcolor{QuartoInternalColor1}{TypeError}\textcolor{QuartoInternalColor2}{: can't multiply sequence by non-int of type 'float'}
\end{Highlighting}

Mit \texttt{asserts} wird überprüft, ob der Wert eines Strings einem
vorgegebenen Wert entspricht und wird für Tests verwendet. Stimmmt der
Wert überein, passiert nichts, ansonsten wird eine Fehlermeldung
ausgegeben.

\begin{Shaded}
\begin{Highlighting}[numbers=left,,]
\ControlFlowTok{assert} \DecValTok{0}\OperatorTok{*} \StringTok{\textquotesingle{}egg\textquotesingle{}} \OperatorTok{==} \StringTok{""}
\end{Highlighting}
\end{Shaded}

\begin{Shaded}
\begin{Highlighting}[numbers=left,,]
\ControlFlowTok{assert} \DecValTok{0}\OperatorTok{*} \StringTok{\textquotesingle{}egg\textquotesingle{}} \OperatorTok{==} \StringTok{\textquotesingle{}egg\textquotesingle{}}
\end{Highlighting}
\end{Shaded}

\begin{Highlighting}
\textcolor{black}{AssertionError: }
\textcolor{black}{}\textcolor{QuartoInternalColor1}{---------------------------------------------------------------------------}\textcolor{QuartoInternalColor2}{}
\textcolor{QuartoInternalColor2}{}\textcolor{QuartoInternalColor1}{AssertionError}\textcolor{QuartoInternalColor2}{                            Traceback (most recent call last)}
\textcolor{QuartoInternalColor2}{}\textcolor{QuartoInternalColor5}{Cell}\textcolor{QuartoInternalColor2}{}\textcolor{QuartoInternalColor5}{ }\textcolor{QuartoInternalColor2}{}\textcolor{QuartoInternalColor3}{In[24]}\textcolor{QuartoInternalColor2}{}\textcolor{QuartoInternalColor3}{, line 1}\textcolor{QuartoInternalColor2}{}
\textcolor{QuartoInternalColor2}{}\textcolor{QuartoInternalColor3}{----> }\textcolor{QuartoInternalColor2}{}\textcolor{QuartoInternalColor3}{1}\textcolor{QuartoInternalColor2}{ }\textcolor{QuartoInternalColor4}{assert}\textcolor{QuartoInternalColor2}{ }\textcolor{QuartoInternalColor3}{0}\textcolor{QuartoInternalColor2}{* }\textcolor{QuartoInternalColor6}{'}\textcolor{QuartoInternalColor2}{}\textcolor{QuartoInternalColor6}{egg}\textcolor{QuartoInternalColor2}{}\textcolor{QuartoInternalColor6}{'}\textcolor{QuartoInternalColor2}{ == }\textcolor{QuartoInternalColor6}{'}\textcolor{QuartoInternalColor2}{}\textcolor{QuartoInternalColor6}{egg}\textcolor{QuartoInternalColor2}{}\textcolor{QuartoInternalColor6}{'}\textcolor{QuartoInternalColor2}{}
\textcolor{QuartoInternalColor2}{}\textcolor{QuartoInternalColor1}{AssertionError}\textcolor{QuartoInternalColor2}{: }
\end{Highlighting}

Python ist Case-Sensitive.

\begin{Shaded}
\begin{Highlighting}[numbers=left,,]
\NormalTok{Number }\OperatorTok{=} \DecValTok{10}
\BuiltInTok{print}\NormalTok{(Number) }
\end{Highlighting}
\end{Shaded}

\begin{verbatim}
10
\end{verbatim}

\begin{Shaded}
\begin{Highlighting}[numbers=left,,]
\BuiltInTok{print}\NormalTok{(number)}
\end{Highlighting}
\end{Shaded}

\begin{Highlighting}
\textcolor{black}{NameError: name 'number' is not defined}
\textcolor{black}{}\textcolor{QuartoInternalColor1}{---------------------------------------------------------------------------}\textcolor{QuartoInternalColor2}{}
\textcolor{QuartoInternalColor2}{}\textcolor{QuartoInternalColor1}{NameError}\textcolor{QuartoInternalColor2}{                                 Traceback (most recent call last)}
\textcolor{QuartoInternalColor2}{}\textcolor{QuartoInternalColor5}{Cell}\textcolor{QuartoInternalColor2}{}\textcolor{QuartoInternalColor5}{ }\textcolor{QuartoInternalColor2}{}\textcolor{QuartoInternalColor3}{In[26]}\textcolor{QuartoInternalColor2}{}\textcolor{QuartoInternalColor3}{, line 1}\textcolor{QuartoInternalColor2}{}
\textcolor{QuartoInternalColor2}{}\textcolor{QuartoInternalColor3}{----> }\textcolor{QuartoInternalColor2}{}\textcolor{QuartoInternalColor3}{1}\textcolor{QuartoInternalColor2}{ }\textcolor{QuartoInternalColor4}{print}\textcolor{QuartoInternalColor2}{(}\textcolor{QuartoInternalColor2}{number}\textcolor{QuartoInternalColor2}{)}
\textcolor{QuartoInternalColor2}{}\textcolor{QuartoInternalColor1}{NameError}\textcolor{QuartoInternalColor2}{: name 'number' is not defined}
\end{Highlighting}

\subsection{Informationstechnische
Betrachtung}\label{informationstechnische-betrachtung}

Der Speicher des Computers lässt sich mit einem Tassenschrank
vergleichen, in dem jede Tasse für eine Variable steht. In jede Tasse
kann ein Wert gelegt werden und der Name der Tasse ist der Bezeichner
der Variablen.

\includegraphics[width=4.16667in,height=\textheight,keepaspectratio]{images/tassenschrank.jpg}

Genauer betrachtet wird bei der Zuweisung von Variablen im Speicher ein
Bereich reserviert, der den Wert der Variablen als Binärzahl speichert.
Der Variablenname (Bezeichner) ist ein daher der Zeiger auf diesen
Bereich.

\includegraphics[width=4.16667in,height=\textheight,keepaspectratio]{images/ref.jpg}

Die Speicheradresse lässt sich mit der id-Funktion ermitteln.

\begin{Shaded}
\begin{Highlighting}[numbers=left,,]
\BuiltInTok{id}\NormalTok{(note)}
\end{Highlighting}
\end{Shaded}

\begin{verbatim}
4379198056
\end{verbatim}

\begin{tcolorbox}[enhanced jigsaw, opacitybacktitle=0.6, left=2mm, coltitle=black, colframe=quarto-callout-tip-color-frame, breakable, colbacktitle=quarto-callout-tip-color!10!white, arc=.35mm, rightrule=.15mm, bottomrule=.15mm, bottomtitle=1mm, colback=white, toptitle=1mm, titlerule=0mm, leftrule=.75mm, toprule=.15mm, title=\textcolor{quarto-callout-tip-color}{\faLightbulb}\hspace{0.5em}{Interpretation der id() Funktion}, opacityback=0]

Die \texttt{id()} Funktion gibt eine \textbf{eindeutige Ganzzahl}
zurück, die das Objekt im Speicher identifiziert.

\textbf{Was bedeutet diese Zahl?} - Die Zahl ist die
\textbf{Speicheradresse} des Objekts (in CPython) - Jedes Objekt hat zur
Laufzeit eine eindeutige ID - Solange das Objekt existiert, bleibt die
ID konstant - Nach dem Löschen des Objekts kann die ID wiederverwendet
werden

\textbf{Beispiel-Interpretation:}

\begin{Shaded}
\begin{Highlighting}[numbers=left,,]
\NormalTok{x }\OperatorTok{=} \DecValTok{42}
\BuiltInTok{print}\NormalTok{(}\BuiltInTok{id}\NormalTok{(x))  }\CommentTok{\# Ausgabe z.B.: 140712345678912}
\end{Highlighting}
\end{Shaded}

Die Zahl \texttt{140712345678912} ist die \textbf{Hexadezimal-Adresse im
Arbeitsspeicher}, wo Python das Objekt mit dem Wert 42 gespeichert hat.

\textbf{Wichtige Erkenntnisse:} - Gleiche ID = Gleiches Objekt im
Speicher - Verschiedene ID = Verschiedene Objekte im Speicher - Die
absolute Zahl ist nicht wichtig, nur der Vergleich zwischen IDs

\end{tcolorbox}

Wird ein weiterer Bezeichner auf die gleiche Variable eingeführt, so
referenziert dieser bei den einfachen Datentypen (int, float) nicht den
gleichen Speicherbereich, sondern erhält eine Wertkopie. Damit ändert
sich der Wert der ersten Variablen nicht, wenn man den Wert der zweiten
Variablen ändert.

\begin{Shaded}
\begin{Highlighting}[numbers=left,,]
\NormalTok{test1 }\OperatorTok{=} \DecValTok{10}
\BuiltInTok{print}\NormalTok{(test1)}
\end{Highlighting}
\end{Shaded}

\begin{verbatim}
10
\end{verbatim}

\begin{Shaded}
\begin{Highlighting}[numbers=left,,]
\NormalTok{test2}\OperatorTok{=}\NormalTok{test1}
\BuiltInTok{print}\NormalTok{(test2)}
\end{Highlighting}
\end{Shaded}

\begin{verbatim}
10
\end{verbatim}

\begin{Shaded}
\begin{Highlighting}[numbers=left,,]
\BuiltInTok{id}\NormalTok{(test1)}
\end{Highlighting}
\end{Shaded}

\begin{verbatim}
4379194440
\end{verbatim}

\begin{Shaded}
\begin{Highlighting}[numbers=left,,]
\BuiltInTok{id}\NormalTok{(test2)}
\end{Highlighting}
\end{Shaded}

\begin{verbatim}
4379194440
\end{verbatim}

\begin{Shaded}
\begin{Highlighting}[numbers=left,,]
\NormalTok{test2}\OperatorTok{=}\DecValTok{9}
\BuiltInTok{print}\NormalTok{(test2)}
\end{Highlighting}
\end{Shaded}

\begin{verbatim}
9
\end{verbatim}

\begin{Shaded}
\begin{Highlighting}[numbers=left,,]
\BuiltInTok{id}\NormalTok{(test2)}
\end{Highlighting}
\end{Shaded}

\begin{verbatim}
4379194408
\end{verbatim}

\begin{Shaded}
\begin{Highlighting}[numbers=left,,]
\BuiltInTok{print}\NormalTok{(test1)}
\end{Highlighting}
\end{Shaded}

\begin{verbatim}
10
\end{verbatim}

\begin{tcolorbox}[enhanced jigsaw, opacitybacktitle=0.6, left=2mm, coltitle=black, colframe=quarto-callout-important-color-frame, breakable, colbacktitle=quarto-callout-important-color!10!white, arc=.35mm, rightrule=.15mm, bottomrule=.15mm, bottomtitle=1mm, colback=white, toptitle=1mm, titlerule=0mm, leftrule=.75mm, toprule=.15mm, title=\textcolor{quarto-callout-important-color}{\faExclamation}\hspace{0.5em}{Erklärung}, opacityback=0]

Python verwendet \textbf{Integer Caching} (Small Integer Optimization)
für kleine Ganzzahlen (typischerweise -5 bis 256). Python speichert die
kleinen Ganzzahlen weil sie häufig verwendet werden, nur einmal im
Speicher und lässt alle Variablen mit demselben Wert auf dasselbe Objekt
zeigen. Sobald Sie test2 einen neuen Wert zuweisen, zeigt es auf ein
anderes gecachtes Objekt und erhält eine neue ID, während test1
weiterhin auf das ursprüngliche Objekt für den Wert 10 zeigt.

Om Detail bedeudet dies:

\begin{itemize}
\tightlist
\item
  Bei \texttt{test1\ =\ 10} wird das gecachte Objekt für 10 verwendet
\item
  Bei \texttt{test2\ =\ test1} zeigt test2 auf dasselbe gecachte Objekt
  → gleiche ID
\item
  Bei \texttt{test2\ =\ 9} wird test2 auf das gecachte Objekt für 9
  umgeleitet → neue ID
\end{itemize}

\textbf{Wichtig:} Obwohl beide Variablen zunächst die gleiche ID haben,
sind sie konzeptionell unabhängig. Eine Änderung von test2 beeinflusst
test1 nicht, da bei der Neuzuweisung ein anderes Objekt referenziert
wird.

Dies ist anders als bei komplexeren Datentypen wie Listen, wo echte
Referenzen verwendet werden.

\end{tcolorbox}

Und wie sieht das bei Fließkommazahlen aus?

\begin{Shaded}
\begin{Highlighting}[numbers=left,,]
\NormalTok{test3 }\OperatorTok{=} \FloatTok{1334.2}
\BuiltInTok{id}\NormalTok{(test3)}
\end{Highlighting}
\end{Shaded}

\begin{verbatim}
4420604848
\end{verbatim}

\begin{Shaded}
\begin{Highlighting}[numbers=left,,]
\NormalTok{test4}\OperatorTok{=}\NormalTok{test3}
\BuiltInTok{id}\NormalTok{(test4)}
\end{Highlighting}
\end{Shaded}

\begin{verbatim}
4420604848
\end{verbatim}

Bei \texttt{float}-Werten gibt es natürlich kein Integer Caching wie bei
kleinen Ganzzahlen. Trotzdem haben \texttt{test3} und \texttt{test4} die
gleiche ID, weil es sich um ein immutable Objekt handelt. Dieses
Verhalten ist bei allen unveränderlichen (immutable) Datentypen in
Python gleich.

\begin{tcolorbox}[enhanced jigsaw, opacitybacktitle=0.6, left=2mm, coltitle=black, colframe=quarto-callout-note-color-frame, breakable, colbacktitle=quarto-callout-note-color!10!white, arc=.35mm, rightrule=.15mm, bottomrule=.15mm, bottomtitle=1mm, colback=white, toptitle=1mm, titlerule=0mm, leftrule=.75mm, toprule=.15mm, title=\textcolor{quarto-callout-note-color}{\faInfo}\hspace{0.5em}{Definition}, opacityback=0]

\textbf{Immutable} (unveränderlich) bedeutet, dass der Wert eines
Objekts nach seiner Erstellung nicht mehr verändert werden kann.

\end{tcolorbox}

\textbf{Immutable Datentypen in Python:} - \texttt{int} (Ganzzahlen) -
\texttt{float} (Fließkommazahlen) - \texttt{str} (Strings/Zeichenketten)
- \texttt{tuple} (Tupel) - \texttt{bool} (Wahrheitswerte) -
\texttt{frozenset} (unveränderliche Mengen)

\textbf{Mutable Datentypen in Python:} - \texttt{list} (Listen) -
\texttt{dict} (Dictionaries/Wörterbücher) - \texttt{set} (Mengen)

\begin{tcolorbox}[enhanced jigsaw, opacitybacktitle=0.6, left=2mm, coltitle=black, colframe=quarto-callout-important-color-frame, breakable, colbacktitle=quarto-callout-important-color!10!white, arc=.35mm, rightrule=.15mm, bottomrule=.15mm, bottomtitle=1mm, colback=white, toptitle=1mm, titlerule=0mm, leftrule=.75mm, toprule=.15mm, title=\textcolor{quarto-callout-important-color}{\faExclamation}\hspace{0.5em}{Wichtig}, opacityback=0]

Bei immutable Datentypen wird bei jeder ``Änderung'' ein neues Objekt
erstellt!

\end{tcolorbox}

Zustand eines Programms nach Ausführung wird vollständig beschrieben
durch die Belegung aller Variablen mit Werten.




\end{document}
