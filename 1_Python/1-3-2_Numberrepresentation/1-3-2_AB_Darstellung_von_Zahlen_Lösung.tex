% Options for packages loaded elsewhere
\PassOptionsToPackage{unicode}{hyperref}
\PassOptionsToPackage{hyphens}{url}
\PassOptionsToPackage{dvipsnames,svgnames,x11names}{xcolor}
%
\documentclass[
  11pt,
  a4paper,
  DIV=11,
  numbers=noendperiod]{scrartcl}

\usepackage{amsmath,amssymb}
\usepackage{iftex}
\ifPDFTeX
  \usepackage[T1]{fontenc}
  \usepackage[utf8]{inputenc}
  \usepackage{textcomp} % provide euro and other symbols
\else % if luatex or xetex
  \usepackage{unicode-math}
  \defaultfontfeatures{Scale=MatchLowercase}
  \defaultfontfeatures[\rmfamily]{Ligatures=TeX,Scale=1}
\fi
\usepackage{lmodern}
\ifPDFTeX\else  
    % xetex/luatex font selection
    \setmainfont[]{Avenir Next}
\fi
% Use upquote if available, for straight quotes in verbatim environments
\IfFileExists{upquote.sty}{\usepackage{upquote}}{}
\IfFileExists{microtype.sty}{% use microtype if available
  \usepackage[]{microtype}
  \UseMicrotypeSet[protrusion]{basicmath} % disable protrusion for tt fonts
}{}
\makeatletter
\@ifundefined{KOMAClassName}{% if non-KOMA class
  \IfFileExists{parskip.sty}{%
    \usepackage{parskip}
  }{% else
    \setlength{\parindent}{0pt}
    \setlength{\parskip}{6pt plus 2pt minus 1pt}}
}{% if KOMA class
  \KOMAoptions{parskip=half}}
\makeatother
\usepackage{xcolor}
\setlength{\emergencystretch}{3em} % prevent overfull lines
\setcounter{secnumdepth}{-\maxdimen} % remove section numbering
% Make \paragraph and \subparagraph free-standing
\makeatletter
\ifx\paragraph\undefined\else
  \let\oldparagraph\paragraph
  \renewcommand{\paragraph}{
    \@ifstar
      \xxxParagraphStar
      \xxxParagraphNoStar
  }
  \newcommand{\xxxParagraphStar}[1]{\oldparagraph*{#1}\mbox{}}
  \newcommand{\xxxParagraphNoStar}[1]{\oldparagraph{#1}\mbox{}}
\fi
\ifx\subparagraph\undefined\else
  \let\oldsubparagraph\subparagraph
  \renewcommand{\subparagraph}{
    \@ifstar
      \xxxSubParagraphStar
      \xxxSubParagraphNoStar
  }
  \newcommand{\xxxSubParagraphStar}[1]{\oldsubparagraph*{#1}\mbox{}}
  \newcommand{\xxxSubParagraphNoStar}[1]{\oldsubparagraph{#1}\mbox{}}
\fi
\makeatother


\providecommand{\tightlist}{%
  \setlength{\itemsep}{0pt}\setlength{\parskip}{0pt}}\usepackage{longtable,booktabs,array}
\usepackage{calc} % for calculating minipage widths
% Correct order of tables after \paragraph or \subparagraph
\usepackage{etoolbox}
\makeatletter
\patchcmd\longtable{\par}{\if@noskipsec\mbox{}\fi\par}{}{}
\makeatother
% Allow footnotes in longtable head/foot
\IfFileExists{footnotehyper.sty}{\usepackage{footnotehyper}}{\usepackage{footnote}}
\makesavenoteenv{longtable}
\usepackage{graphicx}
\makeatletter
\newsavebox\pandoc@box
\newcommand*\pandocbounded[1]{% scales image to fit in text height/width
  \sbox\pandoc@box{#1}%
  \Gscale@div\@tempa{\textheight}{\dimexpr\ht\pandoc@box+\dp\pandoc@box\relax}%
  \Gscale@div\@tempb{\linewidth}{\wd\pandoc@box}%
  \ifdim\@tempb\p@<\@tempa\p@\let\@tempa\@tempb\fi% select the smaller of both
  \ifdim\@tempa\p@<\p@\scalebox{\@tempa}{\usebox\pandoc@box}%
  \else\usebox{\pandoc@box}%
  \fi%
}
% Set default figure placement to htbp
\def\fps@figure{htbp}
\makeatother

\usepackage[document]{ragged2e}
\KOMAoption{captions}{tableheading}
\makeatletter
\@ifpackageloaded{caption}{}{\usepackage{caption}}
\AtBeginDocument{%
\ifdefined\contentsname
  \renewcommand*\contentsname{Table of contents}
\else
  \newcommand\contentsname{Table of contents}
\fi
\ifdefined\listfigurename
  \renewcommand*\listfigurename{List of Figures}
\else
  \newcommand\listfigurename{List of Figures}
\fi
\ifdefined\listtablename
  \renewcommand*\listtablename{List of Tables}
\else
  \newcommand\listtablename{List of Tables}
\fi
\ifdefined\figurename
  \renewcommand*\figurename{Figure}
\else
  \newcommand\figurename{Figure}
\fi
\ifdefined\tablename
  \renewcommand*\tablename{Table}
\else
  \newcommand\tablename{Table}
\fi
}
\@ifpackageloaded{float}{}{\usepackage{float}}
\floatstyle{ruled}
\@ifundefined{c@chapter}{\newfloat{codelisting}{h}{lop}}{\newfloat{codelisting}{h}{lop}[chapter]}
\floatname{codelisting}{Listing}
\newcommand*\listoflistings{\listof{codelisting}{List of Listings}}
\makeatother
\makeatletter
\makeatother
\makeatletter
\@ifpackageloaded{caption}{}{\usepackage{caption}}
\@ifpackageloaded{subcaption}{}{\usepackage{subcaption}}
\makeatother

\usepackage{bookmark}

\IfFileExists{xurl.sty}{\usepackage{xurl}}{} % add URL line breaks if available
\urlstyle{same} % disable monospaced font for URLs
\hypersetup{
  colorlinks=true,
  linkcolor={blue},
  filecolor={Maroon},
  citecolor={Blue},
  urlcolor={Blue},
  pdfcreator={LaTeX via pandoc}}


\author{}
\date{}

\begin{document}


\section{Lösung Aufgabenblatt 3-2}\label{luxf6sung-aufgabenblatt-3-2}

\subsection{Darstellung von Zahlen}\label{darstellung-von-zahlen}

\subsubsection{Aufgabe 1}\label{aufgabe-1}

Berechnen Sie folgende Aufgaben:

\begin{enumerate}
\def\labelenumi{\alph{enumi})}
\item
  \(101010_2 + 1001011_2\)
\item
  \(1011_2 \cdot 110_2\)
\item
  \(1001_2 - 110_2\)
\item
  \(101010 : 10\)
\end{enumerate}

\subsubsection{Lösung}\label{luxf6sung}

\begin{enumerate}
\def\labelenumi{\alph{enumi})}
\tightlist
\item
  \[1110101_2\]
\item
  \[1000010_2\]
\item
  \[11_2\]
\item
  \[ 10101_2\]
\end{enumerate}

\subsubsection{Aufgabe 2}\label{aufgabe-2}

Konvertieren Sie die angebenen Zahlen ins

\begin{enumerate}
\def\labelenumi{\alph{enumi})}
\item
  \(255_{10}\) ins Binärsystem
\item
  \(101010_2\) in Dezimalsystem
\item
  \(122_{10}\) ins Hexadezimalsystem
\item
  \(a10gb_{16}\) ins Dezimalsystem
\end{enumerate}

\subsubsection{Lösung}\label{luxf6sung-1}

\begin{enumerate}
\def\labelenumi{\alph{enumi})}
\tightlist
\item
  \[(11111111)_2\]
\item
  \[(42)_{10}\]
\item
  \[(7A)_{16}\]
\item
  \[ (2576)_{10} \]
\end{enumerate}

\subsubsection{Aufabe 3}\label{aufabe-3}

Handelt es sich bei folgender, umgangssprachlich, prozeduralen
Beschreibung um einen Algorithmus?

Überprüfen Sie dazu die Vorschrift auf die Eigenschaften

\begin{itemize}
\tightlist
\item
  Präszision
\item
  Effektivität
\item
  statische Finitheit
\item
  dynamische Finitheit
\item
  und Terminierung.
\end{itemize}

Begründen Sie kurz ihre Antwort.

\emph{Gegeben seien zwei positive Zahlen a, b. Setze k = 0. Solange k
kleiner ist als b, führe folgende Schritte durch: Addiere b zu a hinzu
(b bleibt unverändert). Falls a nun gerade ist, erhöhe k um 1. Ist k
größer oder gleich b, gib a aus.}

\subsubsection{Lösung}\label{luxf6sung-2}

\textbf{Präzision} ist erfüllt: Jeder Schritt hat eine klare, eindeutige
Interpretation.

\textbf{Effektivität} ist erfüllt: Kein Schritt fordert etwas
Unmögliches wie z.B. eine ganze Zahl zu finden die größer und kleiner 0
gleichzeitig ist.

\textbf{Statische Finitheit} ist erfüllt: Die prozedurale Beschreibung
ist ein endlicher Text.

\textbf{Terminierung} ist nicht generell erfüllt: Ist \texttt{a}
ungerade und \texttt{b} gerade, dann bleibt \texttt{a} stets ungerade,
wodurch \texttt{k} nie erhöht wird und die Bedingung
\texttt{k\ \textless{}\ b} gilt für immer. Das Wiederholen der Schritte
bricht in diesen Fällen also nie ab.

\textbf{Dynamische Finitheit} ist nicht erfüllt: Für alle Eingaben
\texttt{a} und \texttt{b} wird lediglich Speicher für die Werte von
\texttt{a}, \texttt{b} und \texttt{k} benötigt. Da es sich bei den
Werten aber um ganze Zahlen handelt, die beliebig groß werden können,
kann die dynamische Finitheit dennoch verletzt werden. Bei den Eingaben,
die nicht terminieren, wird \texttt{a} mit jedem Schleifendurchlauf
größer, dynamische Finitheit ist also nicht gegeben.

Die prozedurale Beschreibung ist also kein Algorithmus im Sinne der
Vorlesung, da Terminierung und dynamische Finitheit nicht erfüllt sind.

\subsubsection{Aufgabe 4}\label{aufgabe-4}

In einer Werbung wird eine SSD mit 1 Tbyte Speicherplatz beworben.
Nachdem die SSD in den Computer eingebaut wurde, zeigt dieser, dass nur
nur 931 Gbyte Speicher vorhanden sind. Beim Versuch des Umtauschs
behauptet der Verkäufer, dies sei normal. Begründen Sie, warum der
Verkäufer Recht hat.

\subsubsection{Lösung}\label{luxf6sung-3}

Grund ist die unterschiedliche Darstellung der Kapazitäten.
Linuxbasierte Betriessysteme (Linux, MacOS) verwenden das
Dezimalbyte-System, z.B.

1gByte = 1.000.000.000 Byte (Dezimalbytes)

Windows verwendet das Binärbyte-System, d.h. 1 kByte = 1024 Byte, 1
mByte = 1024 kByte 1GB = 1.000.000.000 Byte / 1.024 = 976.562.5 kByte
/1.024 = 953,7 MByte

\subsubsection{Aufgabe 5}\label{aufgabe-5}

\textbf{1. Wahrheitstafel für \(\neg(A \lor B)\):}

\begin{longtable}[]{@{}llll@{}}
\toprule\noalign{}
A & B & \(A \lor B\) & \(\not(A \lor B)\) \\
\midrule\noalign{}
\endhead
\bottomrule\noalign{}
\endlastfoot
0 & 0 & 0 & 1 \\
0 & 1 & 1 & 0 \\
1 & 0 & 1 & 0 \\
1 & 1 & 1 & 0 \\
\end{longtable}

\emph{Ergebnis: Die Negation von \((A \lor B)\) ist nur dann wahr (1),
wenn beide A und B falsch (0) sind.}

\begin{center}\rule{0.5\linewidth}{0.5pt}\end{center}

\textbf{2. Überprüfung von \((A \land B) \Rightarrow A\):}

\begin{longtable}[]{@{}lllll@{}}
\toprule\noalign{}
A & B & \(A \land B\) & \((A \land B) \Rightarrow A\) & Erklärung \\
\midrule\noalign{}
\endhead
\bottomrule\noalign{}
\endlastfoot
0 & 0 & 0 & 1 & 0 → 0 = 1 (wahr) \\
0 & 1 & 0 & 1 & 0 → 0 = 1 (wahr) \\
1 & 0 & 0 & 1 & 0 → 1 = 1 (wahr) \\
1 & 1 & 1 & 1 & 1 → 1 = 1 (wahr) \\
\end{longtable}

\emph{Ergebnis: Die Aussage \((A \land B) \Rightarrow A\) ist eine
\textbf{Tautologie} - sie ist immer wahr.}

\begin{center}\rule{0.5\linewidth}{0.5pt}\end{center}

\textbf{3. Bitweise Multiplikation (AND-Verknüpfung):}

\begin{longtable}[]{@{}llll@{}}
\toprule\noalign{}
Bit A & Bit B & \(A \land B\) & Erklärung \\
\midrule\noalign{}
\endhead
\bottomrule\noalign{}
\endlastfoot
0 & 0 & 0 & \(0 \times 0 = 0\) \\
0 & 1 & 0 & \(0 \times 1 = 0\) \\
1 & 0 & 0 & \(1 \times 0 = 0\) \\
1 & 1 & 1 & \(1 \times 1 = 1\) \\
\end{longtable}

\textbf{Beispiel:} \(1011 \land 1101 = 1001\)

\begin{verbatim}
  1011
∧ 1101
------
  1001
\end{verbatim}

\begin{center}\rule{0.5\linewidth}{0.5pt}\end{center}

\textbf{4. Bitweise Division - Komplexer Algorithmus:}

\emph{Die bitweise Division kann nicht durch eine einfache
Wahrheitstafel dargestellt werden. Sie erfordert einen mehrstufigen
Algorithmus:}

\begin{longtable}[]{@{}lll@{}}
\toprule\noalign{}
Operation & Verwendete Verknüpfungen & Zweck \\
\midrule\noalign{}
\endhead
\bottomrule\noalign{}
\endlastfoot
Vergleich & XOR, AND & Prüfen ob Dividend ≥ Divisor \\
Subtraktion & XOR, AND, NOT & Rest berechnen \\
Verschiebung & Bit-Shift & Position anpassen \\
\end{longtable}

\textbf{Vereinfachtes Beispiel für 1-Bit-Division:} \textbar{} Dividend
\textbar{} Divisor \textbar{} Quotient \textbar{} Rest \textbar{}
\textbar----------\textbar---------\textbar----------\textbar------\textbar{}
\textbar{} 0 \textbar{} 1 \textbar{} 0 \textbar{} 0 \textbar{}
\textbar{} 1 \textbar{} 1 \textbar{} 1 \textbar{} 0 \textbar{}

\begin{center}\rule{0.5\linewidth}{0.5pt}\end{center}

\textbf{5. Bitweise Subtraktion:}

\textbf{Wahrheitstafel für 1-Bit-Subtraktion ohne Borrow:} \textbar{} A
\textbar{} B \textbar{} \(A - B\) \textbar{} Borrow \textbar{}
\textbar---\textbar---\textbar-------\textbar--------\textbar{}
\textbar{} 0 \textbar{} 0 \textbar{} 0 \textbar{} 0 \textbar{}
\textbar{} 0 \textbar{} 1 \textbar{} 1 \textbar{} 1 \textbar{}
\textbar{} 1 \textbar{} 0 \textbar{} 1 \textbar{} 0 \textbar{}
\textbar{} 1 \textbar{} 1 \textbar{} 0 \textbar{} 0 \textbar{}

\textbf{Wahrheitstafel für 1-Bit-Subtraktion mit Borrow\_in:} \textbar{}
A \textbar{} B \textbar{} Borrow\_in \textbar{} Differenz \textbar{}
Borrow\_out \textbar{}
\textbar---\textbar---\textbar-----------\textbar-----------\textbar------------\textbar{}
\textbar{} 0 \textbar{} 0 \textbar{} 0 \textbar{} 0 \textbar{} 0
\textbar{} \textbar{} 0 \textbar{} 0 \textbar{} 1 \textbar{} 1
\textbar{} 1 \textbar{} \textbar{} 0 \textbar{} 1 \textbar{} 0
\textbar{} 1 \textbar{} 1 \textbar{} \textbar{} 0 \textbar{} 1
\textbar{} 1 \textbar{} 0 \textbar{} 1 \textbar{} \textbar{} 1
\textbar{} 0 \textbar{} 0 \textbar{} 1 \textbar{} 0 \textbar{}
\textbar{} 1 \textbar{} 0 \textbar{} 1 \textbar{} 0 \textbar{} 0
\textbar{} \textbar{} 1 \textbar{} 1 \textbar{} 0 \textbar{} 0
\textbar{} 0 \textbar{} \textbar{} 1 \textbar{} 1 \textbar{} 1
\textbar{} 1 \textbar{} 1 \textbar{}

\textbf{Formeln:} - Differenz = A ⊕ B ⊕ Borrow\_in - Borrow\_out = (¬A ∧
B) ∨ (¬A ∧ Borrow\_in) ∨ (B ∧ Borrow\_in)

\textbf{Beispiel:} \(1001 - 0110 = 0011\)

\begin{verbatim}
  1001
- 0110
------
  0011
\end{verbatim}




\end{document}
