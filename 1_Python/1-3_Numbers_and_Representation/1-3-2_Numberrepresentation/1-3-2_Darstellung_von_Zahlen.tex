% Options for packages loaded elsewhere
\PassOptionsToPackage{unicode}{hyperref}
\PassOptionsToPackage{hyphens}{url}
\PassOptionsToPackage{dvipsnames,svgnames,x11names}{xcolor}
%
\documentclass[
  11pt,
  a4paper,
  DIV=11,
  numbers=noendperiod]{scrartcl}

\usepackage{amsmath,amssymb}
\usepackage{iftex}
\ifPDFTeX
  \usepackage[T1]{fontenc}
  \usepackage[utf8]{inputenc}
  \usepackage{textcomp} % provide euro and other symbols
\else % if luatex or xetex
  \usepackage{unicode-math}
  \defaultfontfeatures{Scale=MatchLowercase}
  \defaultfontfeatures[\rmfamily]{Ligatures=TeX,Scale=1}
\fi
\usepackage{lmodern}
\ifPDFTeX\else  
    % xetex/luatex font selection
    \setmainfont[]{Avenir Next}
\fi
% Use upquote if available, for straight quotes in verbatim environments
\IfFileExists{upquote.sty}{\usepackage{upquote}}{}
\IfFileExists{microtype.sty}{% use microtype if available
  \usepackage[]{microtype}
  \UseMicrotypeSet[protrusion]{basicmath} % disable protrusion for tt fonts
}{}
\makeatletter
\@ifundefined{KOMAClassName}{% if non-KOMA class
  \IfFileExists{parskip.sty}{%
    \usepackage{parskip}
  }{% else
    \setlength{\parindent}{0pt}
    \setlength{\parskip}{6pt plus 2pt minus 1pt}}
}{% if KOMA class
  \KOMAoptions{parskip=half}}
\makeatother
\usepackage{xcolor}
\usepackage[top=3cm, bottom=2cm, left=4cm, right=2cm]{geometry}
\setlength{\emergencystretch}{3em} % prevent overfull lines
\setcounter{secnumdepth}{-\maxdimen} % remove section numbering
% Make \paragraph and \subparagraph free-standing
\makeatletter
\ifx\paragraph\undefined\else
  \let\oldparagraph\paragraph
  \renewcommand{\paragraph}{
    \@ifstar
      \xxxParagraphStar
      \xxxParagraphNoStar
  }
  \newcommand{\xxxParagraphStar}[1]{\oldparagraph*{#1}\mbox{}}
  \newcommand{\xxxParagraphNoStar}[1]{\oldparagraph{#1}\mbox{}}
\fi
\ifx\subparagraph\undefined\else
  \let\oldsubparagraph\subparagraph
  \renewcommand{\subparagraph}{
    \@ifstar
      \xxxSubParagraphStar
      \xxxSubParagraphNoStar
  }
  \newcommand{\xxxSubParagraphStar}[1]{\oldsubparagraph*{#1}\mbox{}}
  \newcommand{\xxxSubParagraphNoStar}[1]{\oldsubparagraph{#1}\mbox{}}
\fi
\makeatother


\providecommand{\tightlist}{%
  \setlength{\itemsep}{0pt}\setlength{\parskip}{0pt}}\usepackage{longtable,booktabs,array}
\usepackage{calc} % for calculating minipage widths
% Correct order of tables after \paragraph or \subparagraph
\usepackage{etoolbox}
\makeatletter
\patchcmd\longtable{\par}{\if@noskipsec\mbox{}\fi\par}{}{}
\makeatother
% Allow footnotes in longtable head/foot
\IfFileExists{footnotehyper.sty}{\usepackage{footnotehyper}}{\usepackage{footnote}}
\makesavenoteenv{longtable}
\usepackage{graphicx}
\makeatletter
\newsavebox\pandoc@box
\newcommand*\pandocbounded[1]{% scales image to fit in text height/width
  \sbox\pandoc@box{#1}%
  \Gscale@div\@tempa{\textheight}{\dimexpr\ht\pandoc@box+\dp\pandoc@box\relax}%
  \Gscale@div\@tempb{\linewidth}{\wd\pandoc@box}%
  \ifdim\@tempb\p@<\@tempa\p@\let\@tempa\@tempb\fi% select the smaller of both
  \ifdim\@tempa\p@<\p@\scalebox{\@tempa}{\usebox\pandoc@box}%
  \else\usebox{\pandoc@box}%
  \fi%
}
% Set default figure placement to htbp
\def\fps@figure{htbp}
\makeatother

\usepackage[document]{ragged2e}
\KOMAoption{captions}{tableheading}
\makeatletter
\@ifpackageloaded{tcolorbox}{}{\usepackage[skins,breakable]{tcolorbox}}
\@ifpackageloaded{fontawesome5}{}{\usepackage{fontawesome5}}
\definecolor{quarto-callout-color}{HTML}{909090}
\definecolor{quarto-callout-note-color}{HTML}{0758E5}
\definecolor{quarto-callout-important-color}{HTML}{CC1914}
\definecolor{quarto-callout-warning-color}{HTML}{EB9113}
\definecolor{quarto-callout-tip-color}{HTML}{00A047}
\definecolor{quarto-callout-caution-color}{HTML}{FC5300}
\definecolor{quarto-callout-color-frame}{HTML}{acacac}
\definecolor{quarto-callout-note-color-frame}{HTML}{4582ec}
\definecolor{quarto-callout-important-color-frame}{HTML}{d9534f}
\definecolor{quarto-callout-warning-color-frame}{HTML}{f0ad4e}
\definecolor{quarto-callout-tip-color-frame}{HTML}{02b875}
\definecolor{quarto-callout-caution-color-frame}{HTML}{fd7e14}
\makeatother
\makeatletter
\@ifpackageloaded{caption}{}{\usepackage{caption}}
\AtBeginDocument{%
\ifdefined\contentsname
  \renewcommand*\contentsname{Table of contents}
\else
  \newcommand\contentsname{Table of contents}
\fi
\ifdefined\listfigurename
  \renewcommand*\listfigurename{List of Figures}
\else
  \newcommand\listfigurename{List of Figures}
\fi
\ifdefined\listtablename
  \renewcommand*\listtablename{List of Tables}
\else
  \newcommand\listtablename{List of Tables}
\fi
\ifdefined\figurename
  \renewcommand*\figurename{Figure}
\else
  \newcommand\figurename{Figure}
\fi
\ifdefined\tablename
  \renewcommand*\tablename{Table}
\else
  \newcommand\tablename{Table}
\fi
}
\@ifpackageloaded{float}{}{\usepackage{float}}
\floatstyle{ruled}
\@ifundefined{c@chapter}{\newfloat{codelisting}{h}{lop}}{\newfloat{codelisting}{h}{lop}[chapter]}
\floatname{codelisting}{Listing}
\newcommand*\listoflistings{\listof{codelisting}{List of Listings}}
\makeatother
\makeatletter
\makeatother
\makeatletter
\@ifpackageloaded{caption}{}{\usepackage{caption}}
\@ifpackageloaded{subcaption}{}{\usepackage{subcaption}}
\makeatother

\usepackage{bookmark}

\IfFileExists{xurl.sty}{\usepackage{xurl}}{} % add URL line breaks if available
\urlstyle{same} % disable monospaced font for URLs
\hypersetup{
  pdftitle={3.2 Datenrepräsentation},
  colorlinks=true,
  linkcolor={blue},
  filecolor={Maroon},
  citecolor={Blue},
  urlcolor={Blue},
  pdfcreator={LaTeX via pandoc}}


\title{3.2 Datenrepräsentation}
\author{}
\date{}

\begin{document}
\maketitle


\subsection{3.2.1 Bits}\label{bits}

Ein Bit (Binary Digit) ist die kleinste Informationseinheit in der
digitalen Welt. Es kann nur zwei Zustände annehmen: 0 oder 1. In
Computern werden alle Daten -- Zahlen, Texte, Bilder und Programme --
letztlich als lange Folgen von Bits gespeichert und verarbeitet. Die
Zustände 0 und 1 entsprechen dabei physikalisch unterschiedlichen
Spannungen oder Schaltzuständen in elektronischen Bauteilen.

\begin{tcolorbox}[enhanced jigsaw, opacitybacktitle=0.6, breakable, colframe=quarto-callout-caution-color-frame, arc=.35mm, colbacktitle=quarto-callout-caution-color!10!white, title=\textcolor{quarto-callout-caution-color}{\faFire}\hspace{0.5em}{Achtung}, coltitle=black, leftrule=.75mm, left=2mm, colback=white, rightrule=.15mm, toprule=.15mm, bottomtitle=1mm, toptitle=1mm, opacityback=0, bottomrule=.15mm, titlerule=0mm]

Zur Darstellung eines Bits in Python wird der Datentyp \texttt{bool}
verwendet, der die Werte \texttt{True} (1) und \texttt{False} (0)
repräsentiert.

\end{tcolorbox}

\subsubsection{Grundoperationen auf
Bits}\label{grundoperationen-auf-bits}

Mittels der logischen Interpretation von Bits als Wahrheitswerte, können
drei Grundoperationen definiert werden, die auf Bits angewendet werden
können.

\textbf{1. Logisches AND: \(b_1 \land b_2\)}

\begin{longtable}[]{@{}ccc@{}}
\toprule\noalign{}
\(b_1\) & \(b_2\) & \(b_1 \land b_2\) \\
\midrule\noalign{}
\endhead
\bottomrule\noalign{}
\endlastfoot
0 & 0 & 0 \\
1 & 0 & 0 \\
0 & 1 & 0 \\
1 & 1 & 1 \\
\end{longtable}

\textbf{2. Logisches OR: \(b_1 \lor b_2\)}

\begin{longtable}[]{@{}ccc@{}}
\toprule\noalign{}
\(b_1\) & \(b_2\) & \(b_1 \lor b_2\) \\
\midrule\noalign{}
\endhead
\bottomrule\noalign{}
\endlastfoot
0 & 0 & 0 \\
1 & 0 & 1 \\
0 & 1 & 1 \\
1 & 1 & 1 \\
\end{longtable}

\textbf{3. Logisches NOT (Negation, Komplement): \(\neg b\)}

\begin{longtable}[]{@{}cc@{}}
\toprule\noalign{}
\(b\) & \(\neg b\) \\
\midrule\noalign{}
\endhead
\bottomrule\noalign{}
\endlastfoot
0 & 1 \\
1 & 0 \\
\end{longtable}

\begin{tcolorbox}[enhanced jigsaw, opacitybacktitle=0.6, breakable, colframe=quarto-callout-important-color-frame, arc=.35mm, colbacktitle=quarto-callout-important-color!10!white, title=\textcolor{quarto-callout-important-color}{\faExclamation}\hspace{0.5em}{Satz}, coltitle=black, leftrule=.75mm, left=2mm, colback=white, rightrule=.15mm, toprule=.15mm, bottomtitle=1mm, toptitle=1mm, opacityback=0, bottomrule=.15mm, titlerule=0mm]

Mit den drei Grundoperationen \textbf{AND}, \textbf{OR} und \textbf{NOT}
können alle möglichen Operationen auf Bits definiert werden.

\end{tcolorbox}

\textbf{Beispiel 1}

\begin{longtable}[]{@{}ccc@{}}
\toprule\noalign{}
\(b_1\) & \(b_2\) & \(f_8(b_1, b_2)\) \\
\midrule\noalign{}
\endhead
\bottomrule\noalign{}
\endlastfoot
0 & 0 & 0 \\
1 & 0 & 0 \\
0 & 1 & 0 \\
1 & 1 & 1 \\
\end{longtable}

\textbf{Auflösung:} \(f_8(b_1,b_2)=b_1 \land b_2\)

\textbf{Beispiel 2:}

\begin{longtable}[]{@{}ccc@{}}
\toprule\noalign{}
\(b_1\) & \(b_2\) & \(f_{11}(b_1, b_2)\) \\
\midrule\noalign{}
\endhead
\bottomrule\noalign{}
\endlastfoot
0 & 0 & 1 \\
1 & 0 & 1 \\
0 & 1 & 0 \\
1 & 1 & 1 \\
\end{longtable}

\textbf{Auflösung:} \(f_{11}(b_1,b_2)=\neg b_1 \lor b_2\)

\textbf{Frage:} Wie viele solcher Funktionen mit 2 Bits gibt es?

\textbf{Lösung:} Es gibt \(2^{2^n}\) Funktionen mit n Bits. Für n=2 sind
es also \(2^{2^2}=16\) Funktionen.

\textbf{Bemerkung:}

entweder oder

\begin{longtable}[]{@{}ccc@{}}
\toprule\noalign{}
\(b_1\) & \(b_2\) & \(b_1 \text{ xor }b_2\) \\
\midrule\noalign{}
\endhead
\bottomrule\noalign{}
\endlastfoot
0 & 0 & 0 \\
1 & 0 & 1 \\
0 & 1 & 1 \\
1 & 1 & 0 \\
\end{longtable}

\textbf{Auflösung:}
\(b_1 \text{ xor }b_2=(b_1 \lor b_2) \land \neg(b_1 \land b_2)\)

\textbf{Bemerkung:}

Bitweise Addition mit Übertrag

\begin{longtable}[]{@{}
  >{\centering\arraybackslash}p{(\linewidth - 6\tabcolsep) * \real{0.1282}}
  >{\centering\arraybackslash}p{(\linewidth - 6\tabcolsep) * \real{0.1282}}
  >{\centering\arraybackslash}p{(\linewidth - 6\tabcolsep) * \real{0.5385}}
  >{\raggedright\arraybackslash}p{(\linewidth - 6\tabcolsep) * \real{0.2051}}@{}}
\toprule\noalign{}
\begin{minipage}[b]{\linewidth}\centering
\(b_1\)
\end{minipage} & \begin{minipage}[b]{\linewidth}\centering
\(b_2\)
\end{minipage} & \begin{minipage}[b]{\linewidth}\centering
Summe \(b_1 \text{ xor }b_2\)
\end{minipage} & \begin{minipage}[b]{\linewidth}\raggedright
Übertrag \(b_1 \land b_2\)
\end{minipage} \\
\midrule\noalign{}
\endhead
\bottomrule\noalign{}
\endlastfoot
0 & 0 & 0 & 0 \\
1 & 0 & 1 & 0 \\
0 & 1 & 1 & 0 \\
1 & 1 & 0 & 1 \\
\end{longtable}

\textbf{Bemerkung:}

Bitweise Subtraktion, Multiplikation und Division vergleiche
Arbeitsblatt.

\subsection{Exkurs Aussagenlogik}\label{exkurs-aussagenlogik}

Die Aussagenlogik ist ein grundlegendes Werkzeug, um Sachverhalte
präzise zu beschreiben und logisch zu analysieren. In der Mathematik
wird sie genutzt, um Beweise zu führen, Aussagen zu verknüpfen und die
Gültigkeit von Argumenten zu überprüfen. In der Informatik bildet die
Aussagenlogik die Basis für die Entwicklung von Algorithmen, die
Funktionsweise von Schaltungen, die Programmierung von Bedingungen (if,
while, etc.) und die Fehleranalyse. Sie hilft, komplexe Probleme in
einfache, logisch überprüfbare Einzelschritte zu zerlegen.

\subsubsection{Grundbegriffe}\label{grundbegriffe}

\begin{itemize}
\tightlist
\item
  \textbf{Aussage:} Ein Satz, der eindeutig wahr oder falsch ist.
\item
  \textbf{Wahrheitswert:} Jede Aussage ist entweder wahr (W) oder falsch
  (F).
\item
  \textbf{Logische Operatoren:}

  \begin{itemize}
  \tightlist
  \item
    \textbf{Negation (\(\neg\)):} Nicht
  \item
    \textbf{Konjunktion (\(\land\)):} Und
  \item
    \textbf{Disjunktion (\(\lor\)):} Oder
  \item
    \textbf{Implikation (\(\Rightarrow\)):} Wenn \ldots{} dann
  \item
    \textbf{Äquivalenz (\(\Leftrightarrow\)):} Genau dann, wenn
  \end{itemize}
\end{itemize}

\subsubsection{Wahrheitstafeln}\label{wahrheitstafeln}

\begin{longtable}[]{@{}
  >{\raggedright\arraybackslash}p{(\linewidth - 12\tabcolsep) * \real{0.0769}}
  >{\raggedright\arraybackslash}p{(\linewidth - 12\tabcolsep) * \real{0.0769}}
  >{\raggedright\arraybackslash}p{(\linewidth - 12\tabcolsep) * \real{0.1795}}
  >{\raggedright\arraybackslash}p{(\linewidth - 12\tabcolsep) * \real{0.2821}}
  >{\raggedright\arraybackslash}p{(\linewidth - 12\tabcolsep) * \real{0.1282}}
  >{\raggedright\arraybackslash}p{(\linewidth - 12\tabcolsep) * \real{0.1282}}
  >{\raggedright\arraybackslash}p{(\linewidth - 12\tabcolsep) * \real{0.1282}}@{}}
\toprule\noalign{}
\begin{minipage}[b]{\linewidth}\raggedright
A
\end{minipage} & \begin{minipage}[b]{\linewidth}\raggedright
B
\end{minipage} & \begin{minipage}[b]{\linewidth}\raggedright
\(\neg\) A
\end{minipage} & \begin{minipage}[b]{\linewidth}\raggedright
A \(\land\) B
\end{minipage} & \begin{minipage}[b]{\linewidth}\raggedright
A \(\lor\) B
\end{minipage} & \begin{minipage}[b]{\linewidth}\raggedright
A \(\Rightarrow\) B
\end{minipage} & \begin{minipage}[b]{\linewidth}\raggedright
A \(\Leftrightarrow\) B
\end{minipage} \\
\midrule\noalign{}
\endhead
\bottomrule\noalign{}
\endlastfoot
W & W & F & W & W & W & W \\
W & F & F & F & W & F & F \\
F & W & W & F & W & W & F \\
F & F & W & F & F & W & W \\
\end{longtable}

\subsubsection{Beispiele}\label{beispiele}

\begin{itemize}
\tightlist
\item
  \textbf{Negation:} Die Aussage „Es regnet`` wird zu „Es regnet
  nicht``.
\item
  \textbf{Konjunktion:} „Es regnet und es ist warm.``
\item
  \textbf{Disjunktion:} „Es regnet oder es ist warm.``
\item
  \textbf{Implikation:} „Wenn es regnet, dann ist die Straße nass.``
\item
  \textbf{Äquivalenz:} „Die Lampe leuchtet genau dann, wenn der Schalter
  an ist.``
\end{itemize}

\subsection{3.2.2 Bytes}\label{bytes}

Das Rechnen mit Bits ist ineffizient. Daher werden 8 Bits zu einem Byte
zusammengefasst. Ein Byte dient als Grundeinheit der Datenspeicherung.
Man spricht auch von einem 8-Bitvektor oder von einem Wort der Länge 8.
Es gibt aber auch 16, 32, 64 Bit Worte. Daher spricht man von
Wortbreite. (vgl. 32-Bit- und 64-Bit-Architekturen)

\begin{tcolorbox}[enhanced jigsaw, opacitybacktitle=0.6, breakable, colframe=quarto-callout-important-color-frame, arc=.35mm, colbacktitle=quarto-callout-important-color!10!white, title=\textcolor{quarto-callout-important-color}{\faExclamation}\hspace{0.5em}{Wichtig}, coltitle=black, leftrule=.75mm, left=2mm, colback=white, rightrule=.15mm, toprule=.15mm, bottomtitle=1mm, toptitle=1mm, opacityback=0, bottomrule=.15mm, titlerule=0mm]

Alle Daten werden im Computer als Bitvektoren dargestellt. Die Wortlänge
ist konstant. Die Interpretation des Bitvektors hängt vom Datentyp ab.
(int vs.~float vs string)

\end{tcolorbox}

\subsubsection{Grundoperationen auf
Worten}\label{grundoperationen-auf-worten}

Vergleiche bitweise Operationen auf Bits. Die Operationen werden auf
jedes Bitpaar der beiden Worte an den entsprechenden Stellen angewendet.

\textbf{Beispiel:}

\textbf{Und}

\[\begin{aligned} w_1 \land w_2: \quad 11001010 \land 10000001 &= (1 \land 1)(1 \land 0)(0 \land 0)(0 \land 0)(1 \land 0)(1 \land 0)(0 \land 0)(0 \land 1)\\ &= 10000000\end{aligned}\]

\textbf{Oder}

\[\begin{aligned} w_1 \lor w_2: \quad 11001010 \lor 10000001 &= (1 \lor 1)(1 \lor 0)(0 \lor 0)(0 \lor 0)(1 \lor 0)(1 \lor 0)(0 \lor 0)(0 \lor 1)\\ &= 11001011\end{aligned}\]

\textbf{Negation}

\[\begin{aligned} \neg w: \quad \neg 11001010 &= \neg(1)(1)(0)(0)(1)(0)(1)(0)\\ &= 00110101\end{aligned}\]

\begin{tcolorbox}[enhanced jigsaw, opacitybacktitle=0.6, breakable, colframe=quarto-callout-tip-color-frame, arc=.35mm, colbacktitle=quarto-callout-tip-color!10!white, title=\textcolor{quarto-callout-tip-color}{\faLightbulb}\hspace{0.5em}{Bemerkung}, coltitle=black, leftrule=.75mm, left=2mm, colback=white, rightrule=.15mm, toprule=.15mm, bottomtitle=1mm, toptitle=1mm, opacityback=0, bottomrule=.15mm, titlerule=0mm]

Nur definiert auf Worte gleicher Breite

\end{tcolorbox}

\subsection{3.2.3 Umformen der
Zahldarstellung}\label{umformen-der-zahldarstellung}

Bei uns Menschen herrscht auf Grund der 10 Finger das Dezimalsystem =
Zehnersystem vor. In diesem System gibt es die Ziffern 0, 1, \ldots{} 9.
Möchte man größere Zahlen darstellen, so werden die Ziffern in
verschiedenen Stellenwerten verwendet. Jede Stelle entspricht einer 10er
Potenz, beginnend von rechts mit \(10^0\). Die Basis des Systems ist
somit 10.

\textbf{Beispiel:}

\(3455_{10}=3\cdot 10^3+4\cdot 10^2+5\cdot 10^1+5\cdot 10^0\)

\$ = 3\cdot 1000 + 4 \cdot 100 + 5 \cdot 10 + 5\cdot 1 \$

Ein Computer arbeitet jedoch mit dem Binärsystem, das die kleinste
Informationseinheit ein Bit darstellt, welches den Zustand 0 (Strom aus)
oder Zustand 1 (Strom an) annehmen kann. Die Basis im Binärsystem ist
die 2. Es gibt nur die beiden Ziffern 0 und 1. Möchte man größere Zahlen
darstellen, benötigt man analog zum bekannten Dezimalsystem ein
Stellenwertsystem. Im Fall des Binärsystems wird dafür die Basis 2
verwendet. Jede Stelle einer Zahl im Binärsystem entspricht vom Wert her
einer 2ert Potenz, beginnend von rechts mit \(2^0\).(Gottfried Wilhelm
Leibniz 1646-1716)

\textbf{Beispiel}

\(110011_2 = 1 \cdot 2^5 +1 \cdot 2^4 +0\cdot 2^3+0 \cdot 2^2 + 1 \cdot 2^1 + 1 \cdot 2^0\)

\(= 1 \cdot 32 + 1\cdot16 + 0\cdot 8+ 0\cdot 4+ 1\cdot 2+1\cdot 1\)

Es muss Möglichkeiten geben, um die Zahlendarstellungen umzurechnen. Die
Umrechnung von Binärzahl ind Dezimalzahl ist leicht. Man addiert die
Produkte der Stellenwerte und erhält die Dezimalzahl. Mit der Umrechnung
von Dezimalzahlen in Binärzahlen ist es nicht so einfach.Das obere
Beispiel ergibt somit:

\(110011_2 = 1 \cdot 2^5 +1 \cdot 2^4 +0\cdot 2^3+0 \cdot 2^2 + 1 \cdot 2^1 + 1 \cdot 2^0\)

\(= 1 \cdot 32 + 1\cdot16 + 0\cdot 8+ 0\cdot 4+ 1\cdot 2+1\cdot 1\)

\(= 51_{10}\)

Die Umrechnung von Dezimalzahlen in Binärzahlen ist ein Beispiel für
einen Algorithmus, ein zentrales Konzept der Informatik.

\begin{tcolorbox}[enhanced jigsaw, opacitybacktitle=0.6, breakable, colframe=quarto-callout-note-color-frame, arc=.35mm, colbacktitle=quarto-callout-note-color!10!white, title=\textcolor{quarto-callout-note-color}{\faInfo}\hspace{0.5em}{Definition}, coltitle=black, leftrule=.75mm, left=2mm, colback=white, rightrule=.15mm, toprule=.15mm, bottomtitle=1mm, toptitle=1mm, opacityback=0, bottomrule=.15mm, titlerule=0mm]

Ein Algorithms ist eine Vorschrift mit folgenden Eigenschaften:

\begin{itemize}
\item
  \textbf{Präzision} - Die Bedeutung jedes Schritts ist eindeutig
  festgelegt.
\item
  \textbf{Effektivität} - Jeder Schritt ist ausführbar.
\item
  \textbf{Finitheit (statisch)} - Die Vorschrift ist ein endlicher Text
\item
  \textbf{Finitheit (dynamisch)} - Zur Ausführung wird nur endlich viel
  Speicher benötigt.
\item
  \textbf{Terminierung} - Die Berechnung endet für alle legalen Eingaben
  nach endlich vielen Schritten.
\end{itemize}

Wünschenswert sind dazu folgende Eigenschaften:

\begin{itemize}
\item
  \textbf{Determinismus} - Folgeschritte sind immer eindeutig
  festgelegt.
\item
  \textbf{Determiniertheit} - Bei gleicher Eingabe wird immer die
  gleiche Ausgabe erzeugt.
\item
  \textbf{Generalität} - Die Vorschrift kann eine Klasse von Problemen
  lösen.
\end{itemize}

\end{tcolorbox}

\begin{tcolorbox}[enhanced jigsaw, opacitybacktitle=0.6, breakable, colframe=quarto-callout-tip-color-frame, arc=.35mm, colbacktitle=quarto-callout-tip-color!10!white, title=\textcolor{quarto-callout-tip-color}{\faLightbulb}\hspace{0.5em}{Bemerkung}, coltitle=black, leftrule=.75mm, left=2mm, colback=white, rightrule=.15mm, toprule=.15mm, bottomtitle=1mm, toptitle=1mm, opacityback=0, bottomrule=.15mm, titlerule=0mm]

Auch Vorschriften, welche die wünschenswerten Eigenschaften nicht
erfüllen, werden als Algorithmen angesehen.

\end{tcolorbox}

\textbf{Beispiele:}

\begin{itemize}
\tightlist
\item
  Bedienungsanleitungen
\item
  Aufbauanleitung bei Möbeln zum Beispiel
\item
  Verhaltensvorschriften bei Unfällen, Alarmen, usw.
\item
  Rezepte (Vorsicht: Rezepte enthalten häufig Ungenauigkeiten)
\item
  Rechenvorschriften
\item
  usw.
\end{itemize}

\textbf{Algorithmus zur Umrechnung von Dezimalzahl in Binärzahl:}

Gegeben Dezimalzahl n und die Basis b =2. q ist der Quotient aus n//b
und r der Divisionsrest.

\begin{enumerate}
\def\labelenumi{\arabic{enumi}.}
\tightlist
\item
  Bestimme \(q = n//b\) und \$r = n\%b \$
\item
  Schreibe r links an die Ausgabe
\item
  Falls \(q \neq 0\) gehe zu 1.
\item
  Sonst fertig.
\end{enumerate}

\textbf{Beispiel:}

B=2 und n = 42

\begin{itemize}
\tightlist
\item
  \(42:2=21\) Rest \textbf{0}
\item
  \(21:2=10\) Rest \textbf{1}
\item
  \(10:2=5\) Rest \textbf{0}
\item
  \(5:2=2\) Rest \textbf{1}
\item
  \(2:2=1\) Rest \textbf{0}
\item
  \(1:2=0\) Rest \textbf{1}
\item
  Fertig, weil \(q=0\)
\item
  Ergebnis \(101010_2\) von unten nach oben gelesen.
\end{itemize}

\subsection{3.2.4 Hexadezimalsystem}\label{hexadezimalsystem}

\begin{itemize}
\tightlist
\item
  Stellenwertsystem mit Basis 16 = 4 Bit pro Stelle
\item
  Ziffern: 0, 1, \ldots{} 9, a, b, c, d, e, f
\item
  Jede Stelle entspricht einer 16er Potenz beginnen mit \(16^0\) von
  rechts
\item
  Das Hexadezimalsystem wird häufig in der Informatik verwendet, da es
  eine kompakte Darstellung von Binärzahlen ermöglicht. Jede
  Hexadezimalziffer entspricht genau vier Binärstellen (Bits). Dadurch
  können lange Binärzahlen leichter gelesen und geschrieben werden.
\end{itemize}

\textbf{Beispiel:}

\(\text{beef}_{16} = 11 \cdot 16^3 + 14 \cdot 16^2 +14 \cdot 16^1+ 15 \cdot 16^0\)

\(= 11 \cdot 4096 + 14 \cdot 256 + 14 \cdot 16+ 15\)

\(= 48879_{10}\)

\subsection{3.2.5 Zusammenfassung
Zahlensysteme}\label{zusammenfassung-zahlensysteme}

Welche natürlichen Zahlenbereiche sind mit welcher Wortlänge
darstellbar?

\begin{longtable}[]{@{}ll@{}}
\toprule\noalign{}
Wortlänge (Bits) & Darstellbarer Zahlenbereich \\
\midrule\noalign{}
\endhead
\bottomrule\noalign{}
\endlastfoot
1 & 0 bis 1 \\
2 & 0 bis 3 \\
4 & 0 bis 15 \\
8 & 0 bis 255 \\
16 & 0 bis 65535 \\
32 & 0 bis 4.294.967.295 \\
64 & 0 bis 18.446.744.073.709.551.615 \\
\end{longtable}

\subsubsection{Grundrechenarten von Worten mit mehr als 1
Bit}\label{grundrechenarten-von-worten-mit-mehr-als-1-bit}

\paragraph{Addition}\label{addition}

\begin{longtable}[]{@{}ccc@{}}
\toprule\noalign{}
+ & 0 & 1 \\
\midrule\noalign{}
\endhead
\bottomrule\noalign{}
\endlastfoot
\textbf{0} & 0 & 1 \\
\textbf{1} & 1 & 0 mit Übertrag 1 \\
\end{longtable}

\textbf{Beispiel} \[\begin{aligned}\quad &111001\\
+&111011\\
\text{Übertrag: } &110110\\
\hline
1&110100
\end{aligned}
\]

\paragraph{Subtraktion}\label{subtraktion}

\begin{longtable}[]{@{}ccc@{}}
\toprule\noalign{}
- & \textbf{0} & \textbf{1} \\
\midrule\noalign{}
\endhead
\bottomrule\noalign{}
\endlastfoot
\textbf{0} & 0 & 1 mit Übertrag 1 \\
\textbf{1} & 1 & 0 \\
\end{longtable}

\textbf{Beispiel} \[\begin{aligned}\quad &111001\\
-&111011\\
\text{Übertrag }&110110\\
\hline
1&110100
\end{aligned}
\]

\paragraph{Multiplikation}\label{multiplikation}

\begin{longtable}[]{@{}ccc@{}}
\toprule\noalign{}
* & \textbf{0} & \textbf{1} \\
\midrule\noalign{}
\endhead
\bottomrule\noalign{}
\endlastfoot
\textbf{0} & 0 & 0 \\
\textbf{1} & 0 & 1 \\
\end{longtable}

\textbf{Beispiel} siehe Arbeitsblatt

\paragraph{Division}\label{division}

\begin{longtable}[]{@{}ccc@{}}
\toprule\noalign{}
/ & \textbf{0} & \textbf{1} \\
\midrule\noalign{}
\endhead
\bottomrule\noalign{}
\endlastfoot
\textbf{0} & not defined & 0 \\
\textbf{1} & not defined & 1 \\
\end{longtable}

\textbf{Beispiel} siehe Arbeitsblatt

\subsection{3.2.5 Datentypen - Syntax und
Semantik}\label{datentypen---syntax-und-semantik}

\textbf{1. Semantik eines Datentyps} (Bedeutung)

Menge von Werten und den Operationen auf diesen Werten.

\textbf{2. Syntax} (Schreibweise, Darstellung)

Darstellung eines Wertes (=Literal) und die Operationssymbole zu den
Operationen.

\textbf{3. Pragmatik}

Syntax und Semantik sollen den üblichen mathematischen Konventionen und
Definitionen entsprechen.

\textbf{Beispiele:}

\begin{enumerate}
\def\labelenumi{\arabic{enumi}.}
\tightlist
\item
  Integer
\end{enumerate}

\begin{itemize}
\tightlist
\item
  Die Zahl sechszehn kann durch das Literal 16 dargestellt werden
\item
  Die Zahl sechszehn kann auch durch das Literal 0x10 (hexadezimal)
  dargestellt werden.
\item
  Die Zahl sechszehn kann auch durch das Literals 0b10000 (binär)
  dargestellt werden.
\end{itemize}

\begin{enumerate}
\def\labelenumi{\arabic{enumi}.}
\setcounter{enumi}{1}
\tightlist
\item
  Float
\end{enumerate}

\begin{itemize}
\tightlist
\item
  Die Zahl nullkommazwei kann durch das Literal 0.2 dargestellt werden.
\item
  Die Zahl nullkommazwei kann durch das Literal 2.0e-1 dargestellt
  werden.
\end{itemize}

\begin{enumerate}
\def\labelenumi{\arabic{enumi}.}
\setcounter{enumi}{2}
\tightlist
\item
  String
\end{enumerate}

\begin{itemize}
\tightlist
\item
  Die Zeichenkette \emph{Hund} kann als Literal ``Hund'' dargestellt
  werden.
\item
  Die Zeichenkette \emph{Hund} kann als Literal `Hund' dargestellt
  werden.
\item
  Die Zeichenkette \emph{Hund} kann als Literal '\,`'Hund'\,'\,'
  dargestellt werden.
\end{itemize}

\textbf{In Python}

Jeder Wert besteht aus zwei Teilen:

\begin{enumerate}
\def\labelenumi{\alph{enumi})}
\item
  Typ
\item
  interne Repräsentation des Werts
\end{enumerate}

Die interne Repräsentation des Wertes ist immer eine Folge von Bits.
(Bitvektor) Der Bitvektor wird anhand des Typs interpretiert:

\textbf{0x10} im Datentyp \textbf{int} würde als \textbf{16}
interpretiert werden.

\textbf{0x10} im Datentyp \textbf{float} würde als \textbf{2.24E44}
interpretiert werden.

\textbf{0x40490fd0} im Datentyp \textbf{float} würde als
\textbf{3.14159} interpretiert werden.

\textbf{0x40490fd0} im Datentyp \textbf{int} würde als
\textbf{1078530000} interpretiert werden.

\textbf{0x68656c6c6f00} im Datentyp \textbf{string} würde als
\textbf{hello} interpretiert werden.

\subsection{3.2.6 Realisierung der Ganzzahlwerte im Speicher -
Zwei-Komplementdarstellung}\label{realisierung-der-ganzzahlwerte-im-speicher---zwei-komplementdarstellung}

Zur rechnerinternen Darstellung ganzer Zahlen im 1Bit-Speicher wird
häufig die sogenannte \textbf{2-Komplementdarstellung} verwendet.

\begin{tcolorbox}[enhanced jigsaw, opacitybacktitle=0.6, breakable, colframe=quarto-callout-note-color-frame, arc=.35mm, colbacktitle=quarto-callout-note-color!10!white, title=\textcolor{quarto-callout-note-color}{\faInfo}\hspace{0.5em}{Definition}, coltitle=black, leftrule=.75mm, left=2mm, colback=white, rightrule=.15mm, toprule=.15mm, bottomtitle=1mm, toptitle=1mm, opacityback=0, bottomrule=.15mm, titlerule=0mm]

Ein \textbf{Bit} (=binary digit)ist die kleinste mögliche
Informationseinheit. Mögliche Werte für ein Bit sind 0 oder 1. Dies
entspricht dem Computerzustand Strom an oder Strom aus. In der logischen
Interpretation wäre falsch = 0 und wahr = -1.

\end{tcolorbox}

\begin{tcolorbox}[enhanced jigsaw, opacitybacktitle=0.6, breakable, colframe=quarto-callout-note-color-frame, arc=.35mm, colbacktitle=quarto-callout-note-color!10!white, title=\textcolor{quarto-callout-note-color}{\faInfo}\hspace{0.5em}{Definition}, coltitle=black, leftrule=.75mm, left=2mm, colback=white, rightrule=.15mm, toprule=.15mm, bottomtitle=1mm, toptitle=1mm, opacityback=0, bottomrule=.15mm, titlerule=0mm]

Der Wert einer Zahl \$ a=a\_n a\_\{n-1\} \ldots{} a\_0\$ mit \(n\) Bits
in \textbf{2-Komplementdarstellung} ist definiert als:

\[ \text{Wert}(a) =  -a_n·2^{n}+\sum_{i=0}^{n-1} a_i·2^i \]

\end{tcolorbox}

\textbf{Beispiel: 3-Bit 2-Komplementdarstellung \((n = 3)\)}

\(\text{Wert}(011) = 0\cdot 2^2 + 1 \cdot 2^1 + 1 \cdot 2^0 = 3\)

\(\text{Wert}(101) = -1\cdot 2^2 + 1 \cdot 2^1 + 0 \cdot 2^0 = -3\)

\begin{longtable}[]{@{}
  >{\centering\arraybackslash}p{(\linewidth - 2\tabcolsep) * \real{0.4937}}
  >{\centering\arraybackslash}p{(\linewidth - 2\tabcolsep) * \real{0.5063}}@{}}
\toprule\noalign{}
\begin{minipage}[b]{\linewidth}\centering
Wert in 3-Bit 2-Komplementdarstellung
\end{minipage} & \begin{minipage}[b]{\linewidth}\centering
Wert in Vorzeichen-/Betragsdarstellung
\end{minipage} \\
\midrule\noalign{}
\endhead
\bottomrule\noalign{}
\endlastfoot
000 & 0 \\
001 & 1 \\
010 & 2 \\
011 & 3 \\
100 & -4 \\
101 & -3 \\
110 & -2 \\
111 & -1 \\
\end{longtable}

\textbf{Vorteile}

Vorteilhaft bei dieser Art der Darstellung ist: - Es gibt \textbf{nur
eine Darstellung der Zahl 0}. - Die \textbf{Subtraktion zweier Zahlen}
lässt sich einfach auf eine \textbf{Addition} zurückführen.

\subsection{3.2.7 Realisierung von Fließkommawerten im Speicher -
Gleitkommadarstellung}\label{realisierung-von-flieuxdfkommawerten-im-speicher---gleitkommadarstellung}

\begin{tcolorbox}[enhanced jigsaw, opacitybacktitle=0.6, breakable, colframe=quarto-callout-note-color-frame, arc=.35mm, colbacktitle=quarto-callout-note-color!10!white, title=\textcolor{quarto-callout-note-color}{\faInfo}\hspace{0.5em}{Definition}, coltitle=black, leftrule=.75mm, left=2mm, colback=white, rightrule=.15mm, toprule=.15mm, bottomtitle=1mm, toptitle=1mm, opacityback=0, bottomrule=.15mm, titlerule=0mm]

Eine reelle (eigentlich: rationale) Zahl wird im 1Bit-Speicher
dargestellt durch die sogenannte \textbf{Gleitkommadarstellung}:

\[ z = m \cdot b^e \]

mit - \(m\): Mantisse - \(b\): Basis - \(e\): Exponent (zur Basis b)

In der Praxis wählt man für \(b\) meistens eine der Zahlen 2, 10, 16 und
stellt dementsprechend die Zahl \(m\) als Ziffernfolge im Dual-,
Dezimal-, oder Hexadezimalsystem dar.

\end{tcolorbox}

Der IEEE 754-Standard für eine Zahl \(z\) in Gleitkommadarstellung legt
die Größe der einzelnen Komponenten dieser Darstellung fest und hat
folgendes Format (das erste Bit ist hierbei das Vorzeichen der Mantisse
\(m\)):

32-Bit-IEEE 754-Gleitkommazahlen:

\includegraphics[width=3.125in,height=\textheight,keepaspectratio]{images/floatdarstellung.jpg}

\begin{tcolorbox}[enhanced jigsaw, opacitybacktitle=0.6, breakable, colframe=quarto-callout-tip-color-frame, arc=.35mm, colbacktitle=quarto-callout-tip-color!10!white, title=\textcolor{quarto-callout-tip-color}{\faLightbulb}\hspace{0.5em}{Info}, coltitle=black, leftrule=.75mm, left=2mm, colback=white, rightrule=.15mm, toprule=.15mm, bottomtitle=1mm, toptitle=1mm, opacityback=0, bottomrule=.15mm, titlerule=0mm]

IEEE ist die Abkürzung für ``\textbf{I}nstitute of \textbf{E}lectrical
and \textbf{E}lectronics \textbf{Engineers}'', eine Organisation, die
Standards für verschiedene Technologien entwickelt und pflegt.

\end{tcolorbox}

Bei der Darstellung der Zahlen im Rechner reserviert man einen festen
Teil einer Folge von Speicherzellen (1-Bit Speichern) für die Mantisse
und den Rest für den Exponenten (den Wert von \(b\) braucht man nicht zu
speichern, da alle Rechnungen mit demselben Wert von \(b\) durchgeführt
werden).

\textbf{Beispiel:}

Sei \(b = 2\) und hat man insgesamt 32 Bit zur Verfügung

26 Bit werden für die Mantisse (incl.~Vorzeichen) und 6 Bit für den
Exponenten (incl.~Vorzeichen) 3 Bit verwendet. Das erste Bit
kennzeichnet jeweils das Vorzeichen; dabei kann man insbesondere für den
Exponenten auch die 2-Komplementdarstellung benutzen. Die rechnerinterne
Darstellung der Zahl 12,25 würde dann wie folgt aussehen:

\includegraphics[width=3.125in,height=\textheight,keepaspectratio]{images/beispiel.jpg}

\textbf{Thema zum Weiterdenken:} Die Gleitkommadarstellung ist nicht
eindeutig. Die Zahle 12,26 läst sich auf zum Beispiel diese Arten
darstellen.

\begin{itemize}
\tightlist
\item
  \(1100,01\cdot 2^0\)
\item
  \(110,001\cdot 2^1\)
\item
  \(11,0001\cdot 2^2\)
\item
  usw.
\end{itemize}

Daher wird in der Informatik die \textbf{normierte
Gleitkommadarstellung} verwendet, bei der die Mantisse immer mit einer 1
beginnt (außer bei der Zahl 0). Dadurch wird die Darstellung eindeutig
und es gibt keine Mehrdeutigkeiten mehr. Informiere dich über die
normierte Gleitkommadarstellung.

\begin{tcolorbox}[enhanced jigsaw, opacitybacktitle=0.6, breakable, colframe=quarto-callout-important-color-frame, arc=.35mm, colbacktitle=quarto-callout-important-color!10!white, title=\textcolor{quarto-callout-important-color}{\faExclamation}\hspace{0.5em}{Folgerung}, coltitle=black, leftrule=.75mm, left=2mm, colback=white, rightrule=.15mm, toprule=.15mm, bottomtitle=1mm, toptitle=1mm, opacityback=0, bottomrule=.15mm, titlerule=0mm]

\begin{enumerate}
\def\labelenumi{\arabic{enumi}.}
\tightlist
\item
  Es sind auf diese Weise nur endlich viele reelle Zahlen darstellbar.
\item
  Es gibt jeweils eine kleinste und eine größte darstellbare Zahl.
\item
  Es gibt ein endliches Intevall um den Nullpunkt, in dem keine
  darstellbare Zahl liegt.
\item
  Ein analoges Intervall gibt es selbstverständlich um andere Zahlen.
\end{enumerate}

\end{tcolorbox}




\end{document}
