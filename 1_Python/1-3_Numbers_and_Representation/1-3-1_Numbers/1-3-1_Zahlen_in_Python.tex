% Options for packages loaded elsewhere
\PassOptionsToPackage{unicode}{hyperref}
\PassOptionsToPackage{hyphens}{url}
\PassOptionsToPackage{dvipsnames,svgnames,x11names}{xcolor}
%
\documentclass[
  11pt,
  a4paper,
  DIV=11,
  numbers=noendperiod]{scrartcl}

\usepackage{amsmath,amssymb}
\usepackage{iftex}
\ifPDFTeX
  \usepackage[T1]{fontenc}
  \usepackage[utf8]{inputenc}
  \usepackage{textcomp} % provide euro and other symbols
\else % if luatex or xetex
  \usepackage{unicode-math}
  \defaultfontfeatures{Scale=MatchLowercase}
  \defaultfontfeatures[\rmfamily]{Ligatures=TeX,Scale=1}
\fi
\usepackage{lmodern}
\ifPDFTeX\else  
    % xetex/luatex font selection
    \setmainfont[]{Avenir Next}
\fi
% Use upquote if available, for straight quotes in verbatim environments
\IfFileExists{upquote.sty}{\usepackage{upquote}}{}
\IfFileExists{microtype.sty}{% use microtype if available
  \usepackage[]{microtype}
  \UseMicrotypeSet[protrusion]{basicmath} % disable protrusion for tt fonts
}{}
\makeatletter
\@ifundefined{KOMAClassName}{% if non-KOMA class
  \IfFileExists{parskip.sty}{%
    \usepackage{parskip}
  }{% else
    \setlength{\parindent}{0pt}
    \setlength{\parskip}{6pt plus 2pt minus 1pt}}
}{% if KOMA class
  \KOMAoptions{parskip=half}}
\makeatother
\usepackage{xcolor}
\usepackage[top=3cm, bottom=2cm, left=4cm, right=2cm]{geometry}
\setlength{\emergencystretch}{3em} % prevent overfull lines
\setcounter{secnumdepth}{-\maxdimen} % remove section numbering
% Make \paragraph and \subparagraph free-standing
\makeatletter
\ifx\paragraph\undefined\else
  \let\oldparagraph\paragraph
  \renewcommand{\paragraph}{
    \@ifstar
      \xxxParagraphStar
      \xxxParagraphNoStar
  }
  \newcommand{\xxxParagraphStar}[1]{\oldparagraph*{#1}\mbox{}}
  \newcommand{\xxxParagraphNoStar}[1]{\oldparagraph{#1}\mbox{}}
\fi
\ifx\subparagraph\undefined\else
  \let\oldsubparagraph\subparagraph
  \renewcommand{\subparagraph}{
    \@ifstar
      \xxxSubParagraphStar
      \xxxSubParagraphNoStar
  }
  \newcommand{\xxxSubParagraphStar}[1]{\oldsubparagraph*{#1}\mbox{}}
  \newcommand{\xxxSubParagraphNoStar}[1]{\oldsubparagraph{#1}\mbox{}}
\fi
\makeatother

\usepackage{color}
\usepackage{fancyvrb}
\newcommand{\VerbBar}{|}
\newcommand{\VERB}{\Verb[commandchars=\\\{\}]}
\DefineVerbatimEnvironment{Highlighting}{Verbatim}{commandchars=\\\{\}}
% Add ',fontsize=\small' for more characters per line
\usepackage{framed}
\definecolor{shadecolor}{RGB}{241,243,245}
\newenvironment{Shaded}{\begin{snugshade}}{\end{snugshade}}
\newcommand{\AlertTok}[1]{\textcolor[rgb]{0.68,0.00,0.00}{#1}}
\newcommand{\AnnotationTok}[1]{\textcolor[rgb]{0.37,0.37,0.37}{#1}}
\newcommand{\AttributeTok}[1]{\textcolor[rgb]{0.40,0.45,0.13}{#1}}
\newcommand{\BaseNTok}[1]{\textcolor[rgb]{0.68,0.00,0.00}{#1}}
\newcommand{\BuiltInTok}[1]{\textcolor[rgb]{0.00,0.23,0.31}{#1}}
\newcommand{\CharTok}[1]{\textcolor[rgb]{0.13,0.47,0.30}{#1}}
\newcommand{\CommentTok}[1]{\textcolor[rgb]{0.37,0.37,0.37}{#1}}
\newcommand{\CommentVarTok}[1]{\textcolor[rgb]{0.37,0.37,0.37}{\textit{#1}}}
\newcommand{\ConstantTok}[1]{\textcolor[rgb]{0.56,0.35,0.01}{#1}}
\newcommand{\ControlFlowTok}[1]{\textcolor[rgb]{0.00,0.23,0.31}{\textbf{#1}}}
\newcommand{\DataTypeTok}[1]{\textcolor[rgb]{0.68,0.00,0.00}{#1}}
\newcommand{\DecValTok}[1]{\textcolor[rgb]{0.68,0.00,0.00}{#1}}
\newcommand{\DocumentationTok}[1]{\textcolor[rgb]{0.37,0.37,0.37}{\textit{#1}}}
\newcommand{\ErrorTok}[1]{\textcolor[rgb]{0.68,0.00,0.00}{#1}}
\newcommand{\ExtensionTok}[1]{\textcolor[rgb]{0.00,0.23,0.31}{#1}}
\newcommand{\FloatTok}[1]{\textcolor[rgb]{0.68,0.00,0.00}{#1}}
\newcommand{\FunctionTok}[1]{\textcolor[rgb]{0.28,0.35,0.67}{#1}}
\newcommand{\ImportTok}[1]{\textcolor[rgb]{0.00,0.46,0.62}{#1}}
\newcommand{\InformationTok}[1]{\textcolor[rgb]{0.37,0.37,0.37}{#1}}
\newcommand{\KeywordTok}[1]{\textcolor[rgb]{0.00,0.23,0.31}{\textbf{#1}}}
\newcommand{\NormalTok}[1]{\textcolor[rgb]{0.00,0.23,0.31}{#1}}
\newcommand{\OperatorTok}[1]{\textcolor[rgb]{0.37,0.37,0.37}{#1}}
\newcommand{\OtherTok}[1]{\textcolor[rgb]{0.00,0.23,0.31}{#1}}
\newcommand{\PreprocessorTok}[1]{\textcolor[rgb]{0.68,0.00,0.00}{#1}}
\newcommand{\RegionMarkerTok}[1]{\textcolor[rgb]{0.00,0.23,0.31}{#1}}
\newcommand{\SpecialCharTok}[1]{\textcolor[rgb]{0.37,0.37,0.37}{#1}}
\newcommand{\SpecialStringTok}[1]{\textcolor[rgb]{0.13,0.47,0.30}{#1}}
\newcommand{\StringTok}[1]{\textcolor[rgb]{0.13,0.47,0.30}{#1}}
\newcommand{\VariableTok}[1]{\textcolor[rgb]{0.07,0.07,0.07}{#1}}
\newcommand{\VerbatimStringTok}[1]{\textcolor[rgb]{0.13,0.47,0.30}{#1}}
\newcommand{\WarningTok}[1]{\textcolor[rgb]{0.37,0.37,0.37}{\textit{#1}}}

\providecommand{\tightlist}{%
  \setlength{\itemsep}{0pt}\setlength{\parskip}{0pt}}\usepackage{longtable,booktabs,array}
\usepackage{calc} % for calculating minipage widths
% Correct order of tables after \paragraph or \subparagraph
\usepackage{etoolbox}
\makeatletter
\patchcmd\longtable{\par}{\if@noskipsec\mbox{}\fi\par}{}{}
\makeatother
% Allow footnotes in longtable head/foot
\IfFileExists{footnotehyper.sty}{\usepackage{footnotehyper}}{\usepackage{footnote}}
\makesavenoteenv{longtable}
\usepackage{graphicx}
\makeatletter
\newsavebox\pandoc@box
\newcommand*\pandocbounded[1]{% scales image to fit in text height/width
  \sbox\pandoc@box{#1}%
  \Gscale@div\@tempa{\textheight}{\dimexpr\ht\pandoc@box+\dp\pandoc@box\relax}%
  \Gscale@div\@tempb{\linewidth}{\wd\pandoc@box}%
  \ifdim\@tempb\p@<\@tempa\p@\let\@tempa\@tempb\fi% select the smaller of both
  \ifdim\@tempa\p@<\p@\scalebox{\@tempa}{\usebox\pandoc@box}%
  \else\usebox{\pandoc@box}%
  \fi%
}
% Set default figure placement to htbp
\def\fps@figure{htbp}
\makeatother

\usepackage[document]{ragged2e}
\KOMAoption{captions}{tableheading}
\makeatletter
\@ifpackageloaded{caption}{}{\usepackage{caption}}
\AtBeginDocument{%
\ifdefined\contentsname
  \renewcommand*\contentsname{Table of contents}
\else
  \newcommand\contentsname{Table of contents}
\fi
\ifdefined\listfigurename
  \renewcommand*\listfigurename{List of Figures}
\else
  \newcommand\listfigurename{List of Figures}
\fi
\ifdefined\listtablename
  \renewcommand*\listtablename{List of Tables}
\else
  \newcommand\listtablename{List of Tables}
\fi
\ifdefined\figurename
  \renewcommand*\figurename{Figure}
\else
  \newcommand\figurename{Figure}
\fi
\ifdefined\tablename
  \renewcommand*\tablename{Table}
\else
  \newcommand\tablename{Table}
\fi
}
\@ifpackageloaded{float}{}{\usepackage{float}}
\floatstyle{ruled}
\@ifundefined{c@chapter}{\newfloat{codelisting}{h}{lop}}{\newfloat{codelisting}{h}{lop}[chapter]}
\floatname{codelisting}{Listing}
\newcommand*\listoflistings{\listof{codelisting}{List of Listings}}
\makeatother
\makeatletter
\makeatother
\makeatletter
\@ifpackageloaded{caption}{}{\usepackage{caption}}
\@ifpackageloaded{subcaption}{}{\usepackage{subcaption}}
\makeatother

\usepackage{bookmark}

\IfFileExists{xurl.sty}{\usepackage{xurl}}{} % add URL line breaks if available
\urlstyle{same} % disable monospaced font for URLs
\hypersetup{
  pdftitle={3. Zahlen},
  colorlinks=true,
  linkcolor={blue},
  filecolor={Maroon},
  citecolor={Blue},
  urlcolor={Blue},
  pdfcreator={LaTeX via pandoc}}


\title{3. Zahlen}
\author{}
\date{}

\begin{document}
\maketitle


\subsection{\texorpdfstring{3.1 Ganze Zahlen -
\texttt{int}}{3.1 Ganze Zahlen - int}}\label{ganze-zahlen---int}

Die ganzen Zahlen haben in Python den Datentyp int (=interger =
ganzzahlig)

Für ganze Zahlen gibt es keine Begrenzung, sie können beliebig groß
sein.

Es gibt die bekannten Rechenoperationen wie Addition, Subtraktion,
Multiplikation und Division.

\begin{Shaded}
\begin{Highlighting}[numbers=left,,]
\DecValTok{2}
\end{Highlighting}
\end{Shaded}

\begin{verbatim}
2
\end{verbatim}

\begin{Shaded}
\begin{Highlighting}[numbers=left,,]
\DecValTok{2}\OperatorTok{+}\DecValTok{3}
\end{Highlighting}
\end{Shaded}

\begin{verbatim}
5
\end{verbatim}

\begin{Shaded}
\begin{Highlighting}[numbers=left,,]
\DecValTok{5}\OperatorTok{{-}}\DecValTok{13}
\end{Highlighting}
\end{Shaded}

\begin{verbatim}
-8
\end{verbatim}

\begin{Shaded}
\begin{Highlighting}[numbers=left,,]
\DecValTok{12}\OperatorTok{*}\DecValTok{324}
\end{Highlighting}
\end{Shaded}

\begin{verbatim}
3888
\end{verbatim}

\begin{Shaded}
\begin{Highlighting}[numbers=left,,]
\DecValTok{6}\OperatorTok{/}\DecValTok{3}
\end{Highlighting}
\end{Shaded}

\begin{verbatim}
2.0
\end{verbatim}

\subsection{\texorpdfstring{3.2 Kommazahlen -
\texttt{float}}{3.2 Kommazahlen - float}}\label{kommazahlen---float}

Der Wert der Rechnung 6/3 ist 2.0. Es gibt folglich auch Kommawerte in
python. Diese Zahlen habe den Datentyp \texttt{float} Die Datentypen
sind nicht begrenzt. Auch die Gleitkommazahlen kennen die bekannten
Rechenoperationen wie Addition, Subtraktion, Multiplikation und
Division.

\begin{Shaded}
\begin{Highlighting}[numbers=left,,]
\DecValTok{8}\OperatorTok{/}\DecValTok{4}
\end{Highlighting}
\end{Shaded}

\begin{verbatim}
2.0
\end{verbatim}

\begin{Shaded}
\begin{Highlighting}[numbers=left,,]
\DecValTok{5}\OperatorTok{/}\DecValTok{4}
\end{Highlighting}
\end{Shaded}

\begin{verbatim}
1.25
\end{verbatim}

\begin{Shaded}
\begin{Highlighting}[numbers=left,,]
\DecValTok{7}\OperatorTok{/}\DecValTok{3}
\end{Highlighting}
\end{Shaded}

\begin{verbatim}
2.3333333333333335
\end{verbatim}

\begin{Shaded}
\begin{Highlighting}[numbers=left,,]
\FloatTok{1.2} \OperatorTok{+}\FloatTok{3.34}
\end{Highlighting}
\end{Shaded}

\begin{verbatim}
4.54
\end{verbatim}

\begin{Shaded}
\begin{Highlighting}[numbers=left,,]
\OperatorTok{{-}}\FloatTok{3.45} \OperatorTok{{-}} \FloatTok{7.4}
\end{Highlighting}
\end{Shaded}

\begin{verbatim}
-10.850000000000001
\end{verbatim}

\begin{Shaded}
\begin{Highlighting}[numbers=left,,]
\FloatTok{4.6}\OperatorTok{/}\FloatTok{4.3}
\end{Highlighting}
\end{Shaded}

\begin{verbatim}
1.069767441860465
\end{verbatim}

\begin{Shaded}
\begin{Highlighting}[numbers=left,,]
\OperatorTok{{-}}\FloatTok{2.3}\OperatorTok{*{-}}\FloatTok{3.6}
\end{Highlighting}
\end{Shaded}

\begin{verbatim}
8.28
\end{verbatim}

\begin{Shaded}
\begin{Highlighting}[numbers=left,,]
\DecValTok{1}\OperatorTok{+}\FloatTok{2.3}
\end{Highlighting}
\end{Shaded}

\begin{verbatim}
3.3
\end{verbatim}

\subsection{\texorpdfstring{3.3 Komplexe Zahlen -
\texttt{complex}}{3.3 Komplexe Zahlen - complex}}\label{komplexe-zahlen---complex}

Komplexe Zahlen haben in Python den Datentyp \texttt{complex}. Sie
bestehen aus einem Realteil und einem Imaginärteil, die durch ein
\texttt{j} getrennt sind. Zum Beispiel:

\begin{Shaded}
\begin{Highlighting}[numbers=left,,]
\DecValTok{3}\OperatorTok{+}\OtherTok{4j}
\end{Highlighting}
\end{Shaded}

\begin{verbatim}
(3+4j)
\end{verbatim}

In Python können komplexe Zahlen direkt mit den Operatoren \texttt{+},
\texttt{-}, \texttt{*} und \texttt{/} addiert, subtrahiert,
multipliziert und dividiert werden.

\begin{itemize}
\tightlist
\item
  \textbf{Addition:} \texttt{(a+bj)\ +\ (c+dj)}
\item
  \textbf{Subtraktion:} \texttt{(a+bj)\ -\ (c+dj)}
\item
  \textbf{Multiplikation:} \texttt{(a+bj)\ *\ (c+dj)}
\item
  \textbf{Division:} \texttt{(a+bj)\ /\ (c+dj)}
\end{itemize}

Komplexe Zahlen werden in der Klassenstufe 13.2 eingeführt und im
Studium der Mathematik im Bereich der Funktionentheorie ausführlich
untersucht.

\begin{Shaded}
\begin{Highlighting}[numbers=left,,]
\NormalTok{(}\DecValTok{3}\OperatorTok{+}\OtherTok{4j}\NormalTok{) }\OperatorTok{+}\NormalTok{ (}\OperatorTok{{-}}\DecValTok{2}\OperatorTok{{-}}\OtherTok{5j}\NormalTok{)}
\end{Highlighting}
\end{Shaded}

\begin{verbatim}
(1-1j)
\end{verbatim}

\begin{Shaded}
\begin{Highlighting}[numbers=left,,]
\NormalTok{(}\DecValTok{3}\OperatorTok{+}\OtherTok{4j}\NormalTok{) }\OperatorTok{{-}}\NormalTok{ (}\OperatorTok{{-}}\DecValTok{2}\OperatorTok{{-}}\OtherTok{5j}\NormalTok{)}
\end{Highlighting}
\end{Shaded}

\begin{verbatim}
(5+9j)
\end{verbatim}

\begin{Shaded}
\begin{Highlighting}[numbers=left,,]
\NormalTok{(}\DecValTok{3}\OperatorTok{+}\OtherTok{4j}\NormalTok{) }\OperatorTok{*}\NormalTok{ (}\OperatorTok{{-}}\DecValTok{2}\OperatorTok{{-}}\OtherTok{5j}\NormalTok{)}
\end{Highlighting}
\end{Shaded}

\begin{verbatim}
(14-23j)
\end{verbatim}

\begin{Shaded}
\begin{Highlighting}[numbers=left,,]
\NormalTok{(}\DecValTok{3}\OperatorTok{+}\OtherTok{4j}\NormalTok{) }\OperatorTok{/}\NormalTok{ (}\OperatorTok{{-}}\DecValTok{2}\OperatorTok{{-}}\OtherTok{5j}\NormalTok{)}
\end{Highlighting}
\end{Shaded}

\begin{verbatim}
(-0.896551724137931+0.24137931034482757j)
\end{verbatim}

\subsection{\texorpdfstring{3.4 Bits -
\texttt{bool}}{3.4 Bits - bool}}\label{bits---bool}

Computer kennt zwei Zustände: Strom an und Strom aus. Diese ist also die
kleinste Informationseinheit, die in einem Computer gespeichert werden
kann und wird als binary digit (binär Ziffer), oder kurz Bit,
bezeichnet. In Python gibt es den für ein Bit auch den Datentyp
\texttt{bool}, der zwei mögliche Werte hat: \texttt{True} (wahr, 1) und
\texttt{False} (falsch, 0).

\begin{Shaded}
\begin{Highlighting}[numbers=left,,]
\VariableTok{True}

\end{Highlighting}
\end{Shaded}

\begin{verbatim}
True
\end{verbatim}

\begin{Shaded}
\begin{Highlighting}[numbers=left,,]
\VariableTok{False}
\end{Highlighting}
\end{Shaded}

\begin{verbatim}
False
\end{verbatim}

\begin{Shaded}
\begin{Highlighting}[numbers=left,,]
\DecValTok{5}\OperatorTok{\textgreater{}}\DecValTok{10}
\end{Highlighting}
\end{Shaded}

\begin{verbatim}
False
\end{verbatim}

\begin{Shaded}
\begin{Highlighting}[numbers=left,,]
\DecValTok{5}\OperatorTok{==}\DecValTok{5}
\end{Highlighting}
\end{Shaded}

\begin{verbatim}
True
\end{verbatim}

Auch auf Bits lassen sich die Grundrechenarten anwenden.

\subsubsection{Addition}\label{addition}

\begin{longtable}[]{@{}ccc@{}}
\toprule\noalign{}
+ & 0 & 1 \\
\midrule\noalign{}
\endhead
\bottomrule\noalign{}
\endlastfoot
\textbf{0} & 0 & 1 \\
\textbf{1} & 1 & 0 mit Übertrag 1 \\
\end{longtable}

\subsubsection{Subtraktion}\label{subtraktion}

\begin{longtable}[]{@{}ccc@{}}
\toprule\noalign{}
- & 0 & 1 \\
\midrule\noalign{}
\endhead
\bottomrule\noalign{}
\endlastfoot
\textbf{0} & 0 & 1 mit Übertrag 1 \\
\textbf{1} & 1 & 0 \\
\end{longtable}

\subsubsection{Multiplikation}\label{multiplikation}

\begin{longtable}[]{@{}ccc@{}}
\toprule\noalign{}
* & 0 & 1 \\
\midrule\noalign{}
\endhead
\bottomrule\noalign{}
\endlastfoot
\textbf{0} & 0 & 0 \\
\textbf{1} & 0 & 1 \\
\end{longtable}

\subsubsection{Division}\label{division}

\begin{longtable}[]{@{}ccc@{}}
\toprule\noalign{}
/ & 0 & 1 \\
\midrule\noalign{}
\endhead
\bottomrule\noalign{}
\endlastfoot
\textbf{0} & not defined & 0 \\
\textbf{1} & not defined & 1 \\
\end{longtable}




\end{document}
